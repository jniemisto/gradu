\section{Toiminnallinen testaus}

Robotium vs. monkeyrunner

\subsection{Robotium}

Robotium ei tule Androidin sdk:n mukana, mutta se on paljon käytetty testityökalu Android-sovellusten testauksessa. Robotiumin slogan on, että se on kuin Selenium, mutta Androidille. Selenium taas on laajasti integraatio- ja funktionaalisessa testauksessa käytetty työkalu, joka mahdollistaa selaimen toimintojen automatisoimisen, kuten linkkien klikkauksen, lomakekenttien täyttämisen jne. \cite{selenium}

Robotium on tarkoitettu Android-sovellusten funktionaaliseen-, systeemi- ja hyväksyntätestaukseen. Se on black box -työkalu, eli testin kirjoittajan ei tarvitse päästä käsiksi tai tuntea testattavan sovelluksen koodia. Robotium-testit voivat testata samassa testitapauksessa useita aktiviteetteja. Robotium-testeissä annetaan ohjeita, missä järjestyksessä käyttöliittymäelementtejä klikataan tai syötetään tekstiä.

Robotiumtestejä voi ajaa niin emulaattorissa kuin puhelimessakin. Testit eivät kuitenkaan voi käsitellä kahta eri sovellusta, eli yksi testitapaus voi käsitellä vain yhtä sovellusta. Tällöin sovellustenvälinen integraatiotestaus ei ole mahdollista.

Robotiumin sivuilla sille esitellään useita vahvuuksia Android SDK:n mukana tuleviin työkaluihin verrattuna. Testit vaativat vain vähäistä tuntemusta testattavasta sovelluksesta, Robotium tukee usean aktiviteetin testaamiseta samassa testissä, testien kirjoittamisen nopeus, testikoodin selkeys ja sitkeys, joka johtuu ajoaikaisesta sidonnasta käyttöliittymäkomponentteihin, nopea suoritusnopeus ja helppo integrointi jatkuvan integroinnin työkaluihin Antin tai Mavenin avulla. (pitäisikö ant/maven esitellä jossain?) \cite{robotium}

\subsection{Troyd}

Troyd on Robotiumia käyttäen tehty integraatiotestaustyökalu, jonka tavoite on yhdistää Monkeyn skriptausominaisuudet ja Robotiumin tarjoama korkean tason API. Troyd-testit käyttävät korkean tason komentoja, kuten paina nappia nimeltä x, tarkista, että ruudulla näkyy teksti y, jne, joten testien kirjoituksen pitäisi olla nopeaa. Lisäksi Troyd tarjoaa nauhoitus-toiminnon, jolla testiä voidaan kirjoittaa siten, että testiä kirjoittaessa ohjelma etenee aina seuraavaan tilaan testin mukaisesti. Lopuksi testi tallentuu testitapauksiksi. \cite{troyd}

Troyd-testejä kirjoitetaan Rubylla käyttäen Rubyn Test::Unit-työkalua, joka on Rubyn standardi yksikkötestityökalu. \cite{testunit} Troydin komennot sisältävä TroydCommands-moduli sisällytetään testiluokkaan käyttämällä Rubyn mixin-toiminnallisuutta. Testitapauksia voi kirjoittaa kuten tavallisia test::unit-testejä tai sitten voi käyttää rec-skriptin nauhoitusmahdollisuutta.

Troydin heikkouksia on Jeonin ja Fosterin mielestä mahdollisuus testata vain yhtä sovellusta kerrallaan. Esimerkiksi, jos sovellus aukaisee selainikkunan, Troyd menettää sovelluksen kontrollin. Tämä johtuu Androidin testi-instrumentaation rajoituksista. Toinen Troydin heikkous on hidas suoritusnopeus, koska testiskripti odottaa jokaisen komennon jälkeen, että sovellus on oikeassa tilassa ennen testin jatkamista. \cite{troyd}

\subsection{TEMA}

Pleh.

\subsection{Aiempaa tutkimusta}

Jeon \& Foster mainitsevat Robotiumin vahvuudeksi Androidin omaa Instrumentatiota rikkaamman APIn. Esimerkiksi nappien painamiseen voidaan käyttää nappien nimeä, josta Robotium laskee napin sijainnin. He myös vertaavat Robotiumia omaan Troyd-työkaluunsa ja sanovat sen heikkoudeksi, että testit pitää määritellä etukäteen, eikä niitä pysty muokkaamaan ajonaikaisesti. Muulta toiminnallisuudeltaan Troyd ja Robotium ovat suunnilleen samankaltaisia, koska Troyd on tehty Robotiumin päälle. \cite{troyd}

\subsection{Asennukset ja bleh}

\begin{lstlisting}[float,label=robotium_setup,caption=Robotium testirunko]
public class RobotiumTest extends ActivityInstrumentationTestCase2<Tomdroid> {

	private Solo solo;
	
	public RobotiumTest() {
		super(Tomdroid.class);
	}
	
	@Override
	public void setUp() {
		solo = new Solo(getInstrumentation(), getActivity());
	}
	
	@Override
	public void tearDown() throws Exception {
		solo.finishOpenedActivities();
	}
}
\end{lstlisting}

Robotiumin asennus on yksinkertaista. Projektin build pathiin tarvitsee vain lisätä robotiumin jar-paketti, jossa tulee kaikki tarvittava mukana. Itse robotium-testit perivät Androidin omasta ActivityInstrumentationTestCase2-yliluokasta. Listauksessa \ref{robotium_setup} on esitetty Robotium-testin runko ilman varsinaisia testejä. setUp()-metodissa alustetaan Solo, joka on Robotiumin testit suorittava olio. Se ottaa konstruktoriparametreina ActivityInstrumentationTestCase2:n tarjoaman instrumentaation ja testattavan aktiviteetin. tearDown()-metodissa kutsutaan finishOpenedActivities()-metodia, joka lopettaa kaikki testin aikana aktiivisena olleet aktiviteetit.

\begin{lstlisting}[float,label=delete_notes,caption=Muistikirjojen poisto]
private void removeAllNotes() {
	NoteManager.deleteAllNotes(getActivity());
}
\end{lstlisting}

Robotium-testeissä on huomattava, että jos sovellus muuttaa muistikortille tai muualle tallennettua tilaansa, on testeissä manuaalisesti pidettävä huolta, että sovellus resetoidaan takaisin alkuperäiseen tilaan, jotta testejä voi toistaa useita kertoja luotettavasti. Tomdroidin tapauksessa tämä tehtäisiin kutsumalla tearDown():ssa apumetodia removeAllNotes(), joka kutsuu Tomdroidin NoteManagerin deleteAllNotes()-metodia, joka poistaa kaikki sovelluksen tallentamat muistikirjat.

\subsection{Robotium-testit}

\begin{lstlisting}[float,label=robotium_createnote,caption=Muistikirjan luontitesti robotiumilla]
public void testCreateNoteAddsNote() {
	solo.assertCurrentActivity("Testi alkoi väärästä aktiviteetista", Tomdroid.class);
	assertFalse(solo.searchText("new note"));
	solo.clickOnActionBarItem(R.id.menuNew);
	solo.assertCurrentActivity("Uuden muistikirjan luonti ei avannut uutta muistikirjaa editointinäkymään", EditNote.class);
	solo.enterText(0, "new note");
	solo.clickOnActionBarItem(R.id.edit_note_save);
	solo.clickOnActionBarHomeButton();
	solo.assertCurrentActivity("Koti-näppäimen painaminen ei vienyt takaisin muistikirjalistaan", Tomdroid.class);
	assertTrue(solo.searchText("new note"));
}
\end{lstlisting}

Robotium-testi, jossa testataan uuden muistikirjan luonti, on esitetty listauksessa \ref{robotium_createnote}. Robotiumilla testiä ohjataan Solo-luokan instanssin kautta, jossa on sovelluksen kanssa kommunikointiin tarkoitettuja metodeja, sovelluksen tilasta kertovia metodeja, sekä assertteja. Testin ensimmäisellä rivillä käytetään assertCurrentActivity()-metodia asserttia varmistamaan, että testi alkaa muistikirjalistasta. Toisella rivillä varmistetaan, että testissä luotavaa muistikirjaa ei vielä löydy listasta. Ilman tätä testissä ei voisi olla varma, että muistikirja on luotu onnistuneesti juuri testin aikana. Seuraavalla rivillä painetaan yläpalkin uuden muistikirjan luovaa nappia clickOnActionBarItem()-metodilla. Se ottaa parametrina komponentin id:n, johon ollaan painamassa. Tämän jälkeen pitäisi avautua uusi muistikirja editointinäkymään, mikä varmistetaan seuraavalla rivillä. Sitten syötetään enterText()-metodilla uuden muistikirjan otsikoksi \"new note\". Ensimmäinen parametri kertoo, monenteenko ruudulla näkyvään tekstinmuokkauskomponenttiin teksti syötetään. Tämän jälkeen klikataan yläpalkin tallennus-nappia ja sitten muistikirjalistaukseen vievää nappia. Lopuksi vielä varmistetaan, että palattiin takaisin muistikirjalistaan ja listasta löytyy nyt juuri luotu aktiviteetti.

\subsection{Analyysi}

Se.
\section{Johdanto}

Tällä hetkellä copypaste suoraan aihe-esittelystä.. Refaktoroin sitten kun muu sisältö alkaa olla paremmin hahmotettuna

Googlen kehittämä Android on noussut viime vuosina markkinaosuudeltaan suurimmaksi mobiililaitteiden käyttöjärjestelmäksi. Kuka tahansa voi kehittää Androidille sovelluksia, joiden kehittämiseen tarvittavat välineet ovat ilmaiseksi saatavilla. Erilaisia sovelluksia onkin kehitetty jo noin 500 000.

Mobiilisovellusten kehittämiseen liittyy monia haasteita. Niitä ovat muunmuassa rajalliset laitteistoresurssit, käytettävyys pienellä näytöllä sekä yksityisyyteen ja tietoturvaan liittyvät kysymykset. Androidilla on lisäksi esimerkiksi Applen iOS-alustaan verrattuna omana haasteenaan Android-laitteiden valtava kirjo. Android-laitteita valmistavat kymmenet eri valmistajat ja ne vaihtelevat kameroista tablettien kautta digibokseihin. Laitteissa on hyvin eritehoisia prosessoreita ja lisälaitteita, kuten gps- tai kiihtyvyysantureita, on vaihtelevasti.

Sovellusten laatu on erityisen tärkeää Android-alustalla, jossa kilpailua on runsaasti ja sovellusten hinta niin alhainen, ettei se muodosta estettä sovelluksen vaihtamista toiseen. Sovelluskauppa on myös aina saatavilla suoraan laitteesta.

Gradun tavoitteena on syventyä Android-sovellusten testaamiseen ja testaustyökaluihin. Tutustun kirjallisuudessa esiteltyihin sekä Androidin kehitystyökalujen mukana tuleviin testaustyökaluihin. Kirjallisuuskatsauksen lisäksi vertailen työkaluja testaamalla niiden avulla esimerkkisovellusta ja analysoin testien perusteella niiden puutteita ja vahvuuksia.

Käsittelen gradussa vain Androidille javalla kehitettyjä natiivi-sovelluksia. Androidille voi kehittää sovelluksia myös muunmuassa html5:llä ja erilaisilla työkaluilla, jotka generoivat automaattisesti natiivikoodia useille mobiilialustoille. Osa testaustyökaluista mahdollistaa toki myös tällaisten sovellusten testaamisen. Keskityn vain automaattisiin testaustyökaluihin, joten esimerkiksi manuaaliseen käytettävyystestaukseen liittyvät prosessit ja työkalut on rajattu tämän työn ulkopuolelle.

Aihe-esittelyssä listatut alustavat lähteet: 
\cite{takalaetal11} 
\cite{hyungkeunetal11}
\cite{maetal11}
\cite{spataru10}
\cite{kropp10}
\cite{mirzaeietal12}
\cite{hu10}
\cite{wasserman10}
\cite{hampark11}
\section{Testaamisesta, testityökalujen arvioinnista ja mobiilisovellusten testaamisen erityispiirteistä}

Jotain

\subsection{Testaamisen peruskäsitteitä}

Testitapaus (test case) on yksittäinen testi, jolle on määritelty syötteet, suoritusehdot ja läpäisykriteerit. Testisarja (test suite) taas on joukko testitapauksia. Testisarja voi myös koostua useasta testisarjasta, jolloin esimerkiksi ohjelman jokaiselle komponentille voi olla oma testisarjansa ja yksi testisarja kattaa sitten kaikki yksittäisten komponenttien testisarjat. \cite[153]{testing}

Yksikkötestauksella (unit testing) tarkoitetaan testejä, joiden kohteena on pienimmät mahdolliset ohjelmakomponentit. Olio-ohjelmoinnissa tämä tarkoittaa useimmiten yksittäistä luokkaa, koska yksittäiset metodit vaikuttavat olion tilaan ja siten metodien toimintaan. \cite[282-286]{testing}. JUnit \cite{junit} on Javan standardi yksikkötestaustyökalu.

Yksikkötestaus on useimmiten valkoinen laatikko -testausta (white box testing / structural testing / glass box testing), jolloin testejä voidaan kirjoittaa ohjelmakoodin perusteella. Testeiltä voidaan vaatia esimerkiksi tiettyä koodikattavuutta, jolloin varmistetaan, että mahdollisimman suuri osa ohjelmakoodista tulee suoritettua testien aikana. \cite[154]{testing}

Toiminnallisessa testauksessa (functional testing) testattavan ohjelman sisäistä rakennetta ei tunneta vaan ollaan kiinnostuttu vain syötteistä ja niitä vastaavista tulosteista. Toiminnallista testausta voidaan kutsua myös musta laatikko -testaukseksi (black box testing) (varmista jostain toisesta lähteestä että nämä ovat oikeasti synonyymi, wikipedian mukaan ei ole vaan functional on black boxin alalaji!) Toiminnallisessa testauksessa ollaan kiinnostuttu ohjelman käyttäjälle näkyvästä toiminnasta ja niitä voidaan tehdä esimerkiksi ohjelman määrittelyn pohjalta. \cite[161-162]{testing}

Mallipohjainen testaus (model based testing) on toiminnallisen testauksen alalaji, jossa testitapaukset luodaan automaattisesti ohjelman spesifikaatiosta tehdystä mallista. Jos ohjelma mallinnetaan formaalilla mallilla, kuten äärellisellä automaatilla tai kieliopilla, voidaan testitapaukset generoida täysin automaattisesti. Semiformaaleilla menetelmillä, kuten luokka- tai oliokaavioilla, mallinnetuista ohjelmista generointi saattaa vaatia manuaalista työtä. \cite[245-250]{testing}

Järjestelmätestauksessa (system testing) testataan koko järjestelmää, se perustuu ohjelman havaittavaan toimintaan ja se on itsenäinen suhteessa ohjelman toteutukseen. Jos ohjelma läpäisee järjestelmätestit, sen voi olettaa olevan vapaa tunnetuista bugeista. \cite[418-421]{testing}

Hyväksyntätestien (acceptance tests) tarkoitus on kertoa, onko ohjelma valmist julkaistavaksi. Hyväksyntätesteissä ei etsitä vikoja ohjelmasta, vaan yritetään varmistaa sen riittävä laatutaso. Hyväksyntätestit ovat usein tilastollisia: niissä mitataan ohjelman luotettavuutta, saavutettavuutta tai häiriötiheyttä. Ongelmana tällaisessa testauksessa on, että vaatii hyvin paljon testiainestoa ennen kuin voidaan olla varmoja ohjelman riittävästä laadusta. Usein hyväksyntätestauksessa käytetään alfa- ja beta-testausta, jolloin ohjelman käyttäjät pääsevät testaamaan ohjelmaa ja raportoimaan sen laadusta. \cite[421-423]{testing}

Mockaus (suomennos?) on tekniikka, joka helpottaa yksikkötestien kirjoittamista. Yksikkötestien ulkoiset riippuvuudet voidaan korvata testeissä kontrolloitavilla mock-olioilla. Tällöin testit ovat vakaampia, koska ulkopuolisten komponenttien muokkaus ei vaikuta testeihin, testin haluttuun lopputulokseen vaikuttava ympäristö on helppo saada haluttuun muotoon. Tällöin testien on helppo testata myös sellaisia olosuhteita, jotka ovat harvinaisia, tai vaikea saada muokattua. Lisäksi mockaamalla voidaan korvata vielä tekemättömät ulkoiset riippuvuudet mock-toteutuksilla. \cite{mocking}

\subsection{Mikä on hyvä testityökalu?}

Androidille on tehty Androidin mukana tulevien testaustyökalujen lisäksi monia muita testaustyökaluja. Nämä työkalut erottautuvat Androidin työkaluista jollain tavalla.

Testaustyökalujen arviointia käsittelevät muunmuassa Poston ja Sexton \cite{poston92}. Hyvän testaustyökalun kriteerit ovat osin kontekstista riippuvia, mutta Poston ja Sexton määrittelevät myös yleisempiä kriteerejä testityökalujen arviointiin. Työkalun toiminnallisten ominaisuuksien arviointi on yleensä kontekstiriippuvaista, mutta usein myös helposti määriteltävissä. Jollain työkalulla pystyy testaamaan asioita, joita toisella ei voi. Ei-toiminnallisia olennaisia ominaisuuksia he luettelevat työkalun tehokkuuden, miten nopeasti sen käytön oppii, miten nopeaa testien tekeminen sillä on ja miten luotettava työkalu on.
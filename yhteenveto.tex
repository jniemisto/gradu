\section{Yhteenveto}

Android-sovellukset koostuvat Android-kohtaisista komponenteista: aktiviteeteista, palveluista ja sisällöntarjoajista, sekä tavallisista Java-luokista. Aktiviteetti kuvaa käyttäjälle näkyvää näkymää. Palvelut taas on tarkoitettu pitkäkestoisten taustaoperaatioiden suoritukseen. Sisällöntarjoajat tarjoavat rajapinnan sovelluksen tarvitsemaan tietoon ja mahdollistavat sovellustenvälisen tietojenvaihdon.

Androidin arkkitehtuuri on vahvasti tapahtumapohjaista. Aktiviteetit ja palvelut toteuttavat takaisinkutsumetodit komponenttien elinkaaren hallintaan. Komponentit eivät ole suoraan yhteydessä toisiinsa vaan niiden välillä kommunikoidaan tapahtumapohjaisilla aikeilla, jotka järjestelmä välittää vastaanottavalle komponentille.

Android-sovellusten testaamiseen on kehitetty runsaasti testaustyökaluja. Jo Androidin mukana tulevat työkalut tarjoavat varsin kattavan työkaluvalikoiman Android-sovellusten testaamiseen ohjelmistotuotantoprosessin eri vaiheissa. Tämän lisäksi kolmannen osapuolen kehittämät testaustyökalut, kuten Robolectric ja Robotium, täydentävät Googlen kehittämiä testaustyökaluja.

Tässä tutkielmassa vertailtiin Androidin omia ja kolmansien osapuolien yksikkö- ja toiminnallisen testauksen työkaluja.

Yksikkötestauksessa haasteita tuo se, että vaikka Android-sovelluksia ohjelmoidaan Javalla, Androidin kirjastoluokat eivät toimi suoraan Javan omassa virtuaalikoneessa, vaan testit on ajettava Dalvik-ajoympäristössä emulaattorilla tai Android-laitteessa. Tämä hidastaa testien ajamista. Robolectric-yksikkötestaustyökalu mahdollistaa yksikkötestien ajamisen suoraan Javan virtuaalikoneella. Robolectric osoittautui toimivaksi vaihtoehdoksi Androidin omalle yksikkötestityökalulle. Sen käyttö oli käytännössä yhtä helppoa ja testien ajoaika oli moninkerroin nopeampi kuin Androidin omien yksikkötestien emulaattorissa.

Toiminnallisen testauksen työkaluista vertailtiin Uiautomatoria, Robotiumia ja Troydia. Kukin näistä on toimintatavaltaan hieman erilainen. Uiautomator on Androidin mukana tuleva työkalu, mutta toimii vain Androidin versiolla 4.1 tai uudemmalla. Uiautomator-testien kirjoittaminen on melko verboosia, mutta siihen liittyvä käyttöliittymän analysoiva Uiautomatorviewer tekee testien kirjoittamisesta helppoa. Troyd taas toimii vain Androidin versiolla 2.3.6. Sen erikoisuus on nauhoitusskripti, jonka avulla testiä kirjoitettaessa näkee jatkuvasti sovelluksen tilan testin kussakin vaiheessa. Troyd kuitenkin tarvitsisi runsaasti viimeistelyä, jotta sitä voisi suositella varauksetta. Robotium osoittautui hyväksi toiminnallisen testauksen työkaluksi, joka toimii vakaasti Androidin eri versioilla, joskin testien kirjoittaminen oli hieman Uiautomatoria tai Troydia haastavampaa. Testien ajoaika oli Robotiumilla ja Uiautomatorilla suunnilleen yhtä nopeaa. Troydin testien ajoaika oli reilusti kahta muuta työkalua hitaampi.

Androidin testaustyökalujen tilanne kokonaisuudessaan on varsin hyvä. Google on ottanut testauksen huomioon Androidia kehittäessä ja tarjoaa Androidin kehitystyökalujen mukana melko kattavan paketin erilaisia testityökaluja ohjelmistotuotantoprosessin eri vaiheissa. Tämän lisäksi kolmansien osapuolien kehittämät testaustyökalut täydentävät Androidin omien testaustyökalujen aukkoja.
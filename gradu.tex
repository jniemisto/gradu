\documentclass[emptyfirstpagenumber,gradu]{tktltiki}
\usepackage{ae,aecompl}
\usepackage{url}
\usepackage{amsfonts}
\usepackage{color}
\usepackage{graphicx}
\usepackage{xcolor}
\usepackage{listings}
\lstset{
  literate={ö}{{\"o}}1
           {ä}{{\"a}}1,
  breaklines=true,
  showstringspaces=false,
  basicstyle=\ttfamily\fontsize{11}{12}\selectfont
}
\usepackage{caption}
\DeclareCaptionFont{white}{\color{white}}
\DeclareCaptionFormat{listing}{\colorbox{gray}{\parbox{\textwidth}{#1#2#3}}}
\captionsetup[lstlisting]{format=listing,labelfont=white,textfont=white}
\renewcommand\lstlistingname{Listaus}

\begin{document}
\sloppy
\lstset{language=Java}
\title{Android-sovellusten testaaminen}
\author{Juho Niemistö}
\date{\today}
\level{Pro gradu -tutkielma}
\maketitle

\onehalfspacing

\level{Pro gradu -tutkielma}
\faculty{Matemaattis-luonnontieteellinen}
\department{Tietojenkäsittelytieteen laitos}
\subject{Tietojenkäsittelytiede}
\depositeplace{Kumpulan tiedekirjasto, sarjanumero C-}
\numberofpagesinformation{\numberofpages\ sivua + 10 liitesivua}

\keywords{Android, testaus, yksikkötestaus, toiminnallinen testaus, mobiilisovellukset}

\begin{abstract}
Googlen kehittämä Android on noussut viime vuosina markkinaosuudeltaan suurimmaksi mobiililaitteiden käyttöjärjestelmäksi. Kuka tahansa voi kehittää Androidille sovelluksia, joiden kehittämiseen tarvittavat välineet ovat ilmaiseksi saatavilla. Erilaisia sovelluksia onkin kehitetty jo yli miljoona.

Sovellusten laatu on erityisen tärkeää Android-alustalla, jossa kilpailua on runsaasti ja sovellusten hinta niin alhainen, ettei se muodosta estettä sovelluksen vaihtamiselle toiseen. Sovelluskauppa on myös aina saatavilla suoraan laitteesta. Tämä asettaa sovellusten testaamisellekin haasteita. Toisaalta sovellukset tulisi saada nopeasti sovelluskauppaan, mutta myös sovellusten laadun pitäisi olla hyvä. Testityökalujen pitäisi siis olla helppokäyttöisiä, ja tehokkaita. Androidille onkin kehitetty lukuisia testaustyökaluja Googlen omien työkalujen lisäksi.

Tässä tutkielmassa tutustutaan Android-sovellusten rakenteeseen, niiden testaamiseen ja Android-alustalla toimiviin automaattisen testauksen työkaluihin. Erityisesti keskitytään yksikkö- ja toiminnallisen testauksen työkaluihin. Yksikkötestityökaluista vertaillaan Androidin omaa yksikkötestikehystä Robolectriciin. Toiminnallisen testauksen työkaluista vertaillaan Uiautomatoria, Robotiumia ja Troydia.

ACM Computing Classification System (CCS):
\\* -\textbf{Software and its engineering{\raise.17ex\hbox{$\scriptstyle\mathtt{\sim}$}}Software testing and debugging}
\\* -Software and its engineering{\raise.17ex\hbox{$\scriptstyle\mathtt{\sim}$}}Operating systems
\end{abstract}
\setcounter{tocdepth}{3}
\mytableofcontents


\section{Johdanto}

Googlen kehittämä Android on noussut viime vuosina markkinaosuudeltaan suurimmaksi mobiililaitteiden käyttöjärjestelmäksi. Kuka tahansa voi kehittää Androidille sovelluksia, ja niiden kehittämiseen tarvittavat välineet ovat ilmaiseksi saatavilla. Erilaisia sovelluksia onkin kehitetty jo yli 500 000.

Mobiilisovellusten kehittämiseen liittyy monia haasteita. Niitä ovat muunmuassa rajalliset laitteistoresurssit, käytettävyys pienellä näytöllä sekä yksityisyyteen ja tietoturvaan liittyvät kysymykset. Androidilla on esimerkiksi Applen iOS-alustaan verrattuna omana haasteenaan laitteiden valtava kirjo. Android-laitteita valmistavat kymmenet eri valmistajat ja ne vaihtelevat kameroista tablettien kautta digibokseihin. Laitteissa on hyvin eritehoisia prosessoreita. Myös lisälaitteita, kuten gps- tai kiihtyvyysantureita, on vaihtelevasti.

Sovellusten laatu on erityisen tärkeää Android-alustalla, jossa kilpailua on runsaasti ja sovellusten hinta useimmiten niin alhainen, ettei se muodosta estettä sovelluksen vaihtamiselle toiseen. Sovelluskauppa on myös aina saatavilla suoraan laitteesta. Nämä kilpailutekijät asettavat sovellusten testaamisellekin haasteita. Toisaalta sovellukset tulisi saada nopeasti sovelluskauppaan, mutta myös sovellusten laadun pitäisi olla hyvä. Testityökalujen pitäisi siis olla helppokäyttöisiä ja tehokkaita. Androidille onkin kehitetty useita testaustyökaluja Googlen omien työkalujen lisäksi.

Tutkimuksen tavoitteena on perehtyä Android-sovellusten rakenteeseen, niiden testaamiseen ja Android-alustalla toimiviin automaattisen testauksen työkaluihin. Tutustun kirjallisuudessa esiteltyihin sekä Androidin kehitystyökalujen mukana tuleviin testaustyökaluihin. Kirjallisuuskatsauksen lisäksi vertailen työkaluja testaamalla niiden avulla esimerkkisovellusta ja analysoin testien perusteella niiden puutteita ja vahvuuksia.

Käsittelen tässä tutkielmassa vain Androidille Javalla kehitettyjä natiivisovelluksia. Androidille voi kehittää sovelluksia myös muunmuassa html5:llä ja erilaisilla työkaluilla, jotka generoivat automaattisesti natiivikoodia useille mobiilialustoille. Osa testaustyökaluista mahdollistaa toki myös tällaisten sovellusten testaamisen. Keskityn vain automaattisiin testaustyökaluihin, joten esimerkiksi manuaaliseen käytettävyystestaukseen liittyvät prosessit ja työkalut on rajattu tämän työn ulkopuolelle.

Keskityn yksikkö- ja toiminnallisiin testityökaluihin. Yksikkötestityökaluja käyttävät useimmiten sovelluksen ohjelmoijat itse sovelluksen koodausvaiheessa. Yleistä on myös yksikkötestien kirjoittaminen jo ennen vastaavan ohjelmakoodin kirjoittamista. Toiminnallisissa testeissä testataan sovellusta hieman korkeammalta tasolta ja pyritään varmistamaan haluttujen toiminnallisuuksien toiminta kokonaisuudessaan sovelluksessa. Nämä työkalut olen valinnut huomion kohteeksi, koska toiminnallinen ja yksikkötestaus ovat yleisimmät testauksen muodot mobiilisovellusta kehitettäessä ja työkaluvalikoima on myös monipuolisin.

Toisessa luvussa esittelen Androidin kehitystyökaluja sekä arkkitehtuuria sovellusten kehittäjän näkökulmasta. Luvussa tutustutaan Androidin tärkeimpiin komponentteihin: aktiviteetteihin, palveluihin ja sisällöntarjoajiin sekä komponenttien väliseen kommunikointiin aikeiden avulla. Lisäksi esittelen Android manifestin ja Android-sovellusten julkaisun.

Kolmannessa luvussa käsitellään ohjelmistojen testaamisen perusasioita, testausta osana ohjelmistotuotantoprosessia sekä mobiilisovellusten testaamisen erityispiirteitä. Lisäksi pohditaan testaustyökalujen arviointia ja laatukriteerejä. 

Neljännessä luvussa käsitellään Android-sovellusten testauksen perusteita sekä esitellään Androidin kehitystyökalupaketin mukana tulevia testityökaluja, kuten Monkeyrunner, Monkey ja Uiautomator.

Viidennessä luvussa syvennytään yksikkötestityökaluihin. Vertailen Androidin yksikkötestikehystä avoimen lähdekoodin Robolectriciin, joka pyrkii tarjoamaan Androidin oman yksikkötestikehyksen ominaisuudet Javan omassa virtuaalikoneessa, jotta testien ajaminen olisi nopeampaa. Selvitän luvussa myös yksikkötestikehysten yhteensopivuutta jäljittelijäkehysten kanssa.

Kuudennessa luvussa käsitellään toiminnallisen testauksen työkaluja. Vertailen Androidin kehitystyökalujen mukana tulevan Uiautomator-työkalun lisäksi Robotium ja Troyd -testityökaluja. Kukin näistä työkaluista tarjoaa hieman erilaisen lähestymistavan ja abstraktiotason toiminnallisten testien kirjoittamiseen. Viimeisessä luvussa on yhteenveto.
\clearpage
\section{Android}

Tässä luvussa esitellään Androidin historiaa, kehitystyökaluja sekä Android-sovellusten arkkitehtuuria ja pääkomponentit.

\subsection{Historia}

Androidin kehityksen aloitti Android Inc. -niminen yritys vuonna 2003. Google osti sen vuonna 2005. Kaksi vuotta myöhemmin, marraskuussa 2007 Androidin ensimmäinen versio julkaistiin ja samalla kerrottiin, että sen kehityksestä vastaa Open Handset Alliance, johon kuului Googlen lisäksi puhelinvalmistajia, kuten HTC ja Samsung, operaattoreita, kuten Sprint Nextel ja T-Mobile sekä komponenttivalmistajia, kuten Qualcomm ja Texas Instruments.

Ensimmäinen Androidille julkaistu kaupallinen laite oli HTC Dream -älypuhelin, joka julkaistiin lokakuussa 2008.
Loppuvuodesta 2010 Android nousi älypuhelinten markkinajohtajaksi. Syksyllä 2012 Androidilla oli jo tutkimuksesta riippuen 50-70 prosentin markkinaosuus ja laitevalikoima on kasvanut älypuhelimista muunmuassa tablet-tietokoneisiin, digibokseihin ja kameroihin \cite{wikiandroid}.

\subsection{Androidin kehitystyökalut}

Android-sovelluksia tehdään Java-ohjelmointikielellä. Google julkaisee Androidille ilmaista ohjelmistokehitystyökalua (Android SDK), joka kääntää sovelluksen ja pakkaa sen kuvien ja muiden resurssien kanssa apk-tiedostoksi (Android Application Package). Apk-tiedosto sisältää kaiken yhden sovelluksen asentamiseen tarvittavat tiedot. Android-sovellusten kehittämiseen tarvitsee käytännössä Javan kehitystyökaluista (SDK) version 5 tai 6, Androidin SDK:n, Eclipsen sekä Android-laajennoksen (Android Development Tools, ADT) Eclipselle.

Androidin SDK:n mukana tulee minimoitu versio Androidin järjestelmäkirjastoista, joiden avulla Eclipse osaa opastaa Androidin rajapintojen käytössä. Rajapinnan takana ei ole kuitenkaan oikeaa toteutusta, joten esimerkiksi yksikkötestit, jotka menevät kirjastoluokkiin asti, eivät toimi Eclipsestä Javalla ajettaessa.

Sovelluksen ja testien ajamista varten SDK:n mukana tulee Android-emulaattori. Emulaattoreita voi ajaa eri Android API:n versioilla ja laitteistoprofiileilla, jotta on mahdollista testata sovelluksen toimivuutta erilaisissa Android-ympäristöissä. Emulaattori kykenee myös jossain määrin simuloimaan lisälaitteiden, kuten kiihtyvyysanturin, toimintaa. Suurin puute emulaattorissa on sen heikko suoritusnopeus. Sovelluksia ja testejä voi ajaa myös suoraan tietokoneeseen liitetyssä Android-laitteessa. 

Androidin Eclipse-laajennos toimii siltana Android SDK:n ja Eclipsen välillä mahdollistaen SDK:n tarjoamien ominaisuuksien hyödyntämisen suoraan Eclipsestä käsin. ADT:n avulla on myös mahdollista seurata tietokoneeseen kytkettyjen Android-laitteiden tapahtumalogeja ja debug-tietoja \cite[25-50]{androidgamedev}.

\subsection{Android-sovellusten rakenne}

Android on rakennettu Linuxin ytimen version 2.6 päälle ja koko Androidin järjestelmäkoodi on avointa, mikä tarkoittaa, että mikä tahansa valmistaja voi tehdä Androidin pohjalta oman mobiilikäyttöjärjestelmänsä. Jokainen sovellus on käyttäjänä järjestelmässä. Sovellusten oikeudet on rajattu siten, että ne pääsevät käsiksi vain kyseiseen sovellukseen liittyviin resursseihin. Sovelluksen ollessa käynnissä, se pyörii omana prosessina Linux-prosessien tavoin, jota Android-käyttöjärjestelmä hallitsee. Androidin turvallisuusratkaisu noudattaa vähimmän mahdollisen tiedon periaatetta; sovelluksella on vain ne oikeudet, joita se vähintään tarvitsee toimintaansa. Kaikkia ylimääräisiä oikeuksia varten täytyy erikseen pyytää lupa.

\begin{figure}[htb]
\includegraphics[width=130mm]{class_diagram.png}
\caption{Androidin tärkeimpien komponenttien luokkahierarkia} \label{class_diagram}
\end{figure}

Android-sovellukset koostuvat neljästä komponenttityypistä: aktiviteeteista (activities), palveluista (services), sisällöntarjoajista (content providers) sekä lähetysten vastaanottajista (broadcast receivers). Komponenttien välinen kommunikointi on pääosin tapahtumapohjaista; eri komponentit eivät keskustele suoraan keskenään, vaan kaikki siirtymät komponenttien välillä tapahtuvat käyttöjärjestelmän välittämien tapahtumaviestien perusteella. Tämän vaikutuksesta Android-sovellukset voivat helposti käyttää toiminnassaan järjestelmän ja toisten sovellusten tarjoamia komponentteja.

Kuvassa \ref{class_diagram} on esitelty Androidin peruskomponenttien muodostama luokkahierarkia. Vain aktiviteeteilla ja palveluilla on yhteinen yliluokka Context, lähetysten vastaanottajat ja sisällöntarjoajajat perivät vain Javan geneerisen Object-luokan. Kuvaa on yksinkertaistettu siten, että Contextin ja Activityn ja Servicen välillä olevia Wrapper-luokkia on jätetty kuvaamatta perintähierarkiassa. Context-luokka tarjoaa aktiviteettien ja palveluiden käyttöön sovelluksen globaaliin tilaan liittyviä tietoja.

Aktiviteetti kuvaa yhtä sovelluksen käyttöliittymän kerrallaan muodostavaa näkymää. Sovelluksen käyttöliittymä koostuu useista aktiviteeteista, jotka muodostavat yhtenäisen sovelluksen, mutta jokainen aktiviteetti on toisistaan riippumaton. Eri sovellukset voivat myös käynnistää toistensa aktiviteetteja, mikäli vastaanottava sovellus sen sallii. Esimerkiksi kamera-sovellus voi käynnistää sähköposti-sovelluksen sähköpostinkirjoitus-aktiviteetin, jos ottamansa kuvan haluaa jakaa sähköpostilla. Aktiviteetit ovat Androidissa Activity-luokan aliluokkia.

Palvelut ovat taustaprosesseja, jotka suorittavat pitkäkestoisia operaatioita, kuten tiedon lataamista verkosta tai musiikin soittamista taustalla samalla, kun käyttäjä käyttää toista sovellusta. Palvelut eivät tarjoa käyttöliittymää ja toiset komponentit, kuten aktiviteetit, voivat käynnistää niitä. Palvelut ovat Service-luokan aliluokkia.

Sisällöntarjoajat vastaavat sovelluksen tarvitseman tiedon lukemisesta ja kirjoittamisesta pitkäkestoiseen muistiin. Tallennuspaikkana voi olla laitteen tiedostojärjestelmä, SQLite-tietokanta, verkko tai ylipäänsä mikä tahansa kohde, johon sovelluksella on luku- tai kirjoitusoikeudet. Sovellukset voivat käyttää toistensa sisällöntarjoajia, mikäli sovellus julkaisee ne muiden sovellusten käyttöön. Sisällöntarjoajat ovat ContentProvider-luokan aliluokkia.

Lähetysten vastaanottajat reagoivat järjestelmänlaajuisiin viesteihin ja tapahtumiin. Tällaisia ovat esimerkiksi ilmoitus, että akku on lopussa tai että käyttäjä on sulkenut tai avannut näytön. Ne voivat myös lähettää järjestelmänlaajuisia tapahtumaviestejä muille sovelluksille. Lähetysten vastaanottajat ovat BroadcastReceiver-luokan aliluokkia, ja tapahtumat ovat Intent-luokan aliluokkia.

Android-sovellukset käyttävät usein hyväkseen toisten sovellusten komponentteja. Sovellukset eivät pysty suoraan kutsumaan toisiaan, vaan halutessaan hyödyntää toisten sovellusten ominaisuuksia sovellus luo uuden aikeen, jonka järjestelmä välittää tiettyjen sääntöjen perusteella sopivalle vastaanottajalle (katso luku \ref{intents}). 

Android-sovelluksilla ei ole yksittäistä main-metodia, joka käynnistäisi ohjelman, kuten usein muissa sovelluksissa on tapana. Sovellus voi sen sijaan käynnistyä vastaanottamansa aikeen johdosta monen eri komponentin kautta. Lisäksi sovellus saatetaan joutua käynnistämään ja sulkemaan useita kertoja esimerkiksi käyttäjän vaihtaessa puhelimen orientaatiota tai vastaanotettaessa puhelua, joten sovelluksen pitää pystyä tehokkaasti palautumaan keskeytyneeseen tilaansa. Sovelluksella on siis lukuisia mahdollisia käynnistymis- ja sulkeutumispolkuja. Tämän takia ohjelmakomponenttien elinkaaren hallinta on tärkeä osa sovelluksen rakentamista.

Android-sovelluksilla on xml-muotoinen Manifest-tiedosto, jossa määritellään sovelluksen komponentit, niiden näkyvyys ja minkälaisia tapahtumia ne osaavat hallita. Manifestissa määritellään myös mitä rajoitteita sovellus asettaa käytettävissä olevalle Android-versiolle, puhelimen ominaisuuksille, kuten lisälaitteiden saatavuudelle ja näyttöresoluutiolle, ja mitä oikeuksia sovellus vaatii toimiakseen. Näin voidaan varmistaa, että sovellusta ei asenneta laitteelle, jossa ei ole sovelluksen välttämättä tarvitsemia ominaisuuksia \cite{android}.

\subsection{Aktiviteetit}

\begin{figure}[htb]
\includegraphics[width=100mm]{activity_lifecycle.png}
\caption{Aktiviteetin elinkaari} \label{activity_lifecycle}
\end{figure}

Aktiviteetti kuvaa yhtä sovelluksen käyttöliittymän näkymää. Lähes aina aktiviteetti on koko näytön kokoinen - eli kaikki, mitä puhelimen ruudulla näkyy yläreunan status-palkkia lukuunottamatta on samaa aktiviteettia - mutta ne voivat olla myös pienempiä tai leijua osittain toisen aktiviteetin päällä. Kuitenkin vain yksi aktiviteetti voi olla aktiivinen, eli reagoida käyttäjän syötteisiin, kerrallaan. Yksi sovelluksen aktiviteeteistä on yleensä pääaktiviteetti, joka käynnistyy silloin, kun käyttäjä avaa sovelluksen. 

Aktiviteettien elinkaaren hallinta on Android-sovelluksen kriittisimpiä osia, koska järjestelmän resurssit ovat yleensä hyvin rajalliset ja Android-laitteiden käyttöön liittyy usein tiheä vaihtelu eri sovellusten välillä. Tällöin on tärkeää, että sovellus luovuttaa varaamansa resurssit muiden sovellusten käyttöön, kun sovellus vaihtuu, ja vastaavasti osaa palautua takaisin pysäytettäessä olleeseen tilaan käyttäjän palatessa sovellukseen. Nämä vaihdokset pitäisi lisäksi tapahtua mahdollisimman tehokkaasti, jotta järjestelmän toiminta olisi käyttäjän näkökulmasta mahdollisimman sulavaa sovellusten tilojen vaihtamisen yhteydessä.

Aktiviteetilla voi olla pitkäkestoisemmin kolme eri tilaa. Aktiviteetti on aktiivisessa tilassa (resumed) silloin, kun se on näytön etualalla ja käyttäjä käyttää juuri sitä aktiviteettia. Keskeytetyssä (paused) tilassa aktiviteetti on, kun se on osittain näkyvissä, mutta jokin toinen aktiviteetti on aktiivisena sen päällä. Keskeytetyt aktiviteetit ja niiden tilat pysyvät muistissa, joskin jos laitteen muisti on lopussa, järjestelmä saattaa tuhota sen. Aktiviteetti on pysäytetty (stopped) silloin, kun jokin toinen aktiviteetti peittää sen kokonaan näkyvistä. Tällainenkin aktiviteetti säilyy muistissa, jos laitteen resurssit ovat riittävät, mutta järjestelmä voi tuhota sen koska vain, jos resursseja tarvitaan muiden aktiviteettien käyttöön.

Aktiviteetin siirtyminen eri tilojen välillä tapahtuu järjestelmän kutsuessa aktiviteetin takaisinkutsumetodeita. Mahdolliset tilasiirtymäpolut näkyvät kuvassa \ref{activity_lifecycle}. 

Aktiviteetin koko elinkaari tapahtuu onCreate()- ja onDestroy()-kutsujen välillä. Aktiviteetin tulisi tehdä kaikki kerran suoritettavat tilanalustustehtävät kutsuttaessa onCreate()-metodia, kuten ulkoasun määrittely tai koko aktiviteetin elinkaaren ajan tarvittavan tiedonsiirtosäikeen avaus. Vastaavasti onDestroy()-kutsussa aktiviteetin tulisi vapauttaa kaikki loputkin aktiviteetin varaamat resurssit.

Aktiviteetin käyttäjälle näkyvä elinkaari on onStart()- ja onStop()-kutsujen välillä. onStart()-metodia kutsutaan, kun aktiviteetti tulee näkyväksi käyttäjälle, ja onStop()-metodia kutsutaan, kun jokin toinen aktiviteetti on peittänyt kyseisen aktiviteetin kokonaan. Näkyvän elinkaaren aikana tulisi ylläpitää niitä resursseja, joita tarvitaan käyttäjän kanssa kommunikointiin sekä sellaisia, jotka saattavat muuten vaikuttaa käyttäjälle näkyvään käytöliittymään. Esimerkiksi lähetystenvastaanottajaa on hyvä kuunnella tällä välillä mahdollisten järjestelmänlaajuisten käyttöliittymään vaikuttavien tapahtumien varalta. onStart() ja onStop() -kutsuja voi tulla lukuisia aktiviteetin koko elinkaaren aikana. onRestart()-metodia kutsutaan, jos aktiviteetti on jo luotu aiemmin ja pysäytetty sitten onStop()-kutsulla. onRestart()-kutsua seuraa aina onStart()-kutsu.

Aktiviteetti on aktiivisena näytön etualalla onResume() ja onPause() -kutsujen välillä. Kun aktiviteetti on etualalla, käyttäjä käyttää juuri sitä ja se on kaikkien muiden aktiviteettien päällä. onResume() ja onPause() -kutsuja voi tulla tiheästi, esimerkiksi aina kun laitteen näyttö menee lepotilaan tai tulee jokin ilmoitus aktiviteetin päälle, joten niiden toteutus ei saa olla liian raskas.

Androidin järjestelmä voi tuhota sovelluksen prosessin onPausen(), onStopin() tai onDestroyn() jälkeen. Tämän takia pysyväksi tarkoitettu tieto on tallennettava onPause()-kutsun jälkeen. Tallennus voidaan tehdä esimerkiksi toteuttamalla takaisinkutsumetodi onSaveInstanceData(), jota kutsutaan aina, ennen kuin järjestelmä mahdollistaa aktiviteetin tuhoamisen. onSaveInstanceData() saa parametrinaan Bundle-olion, johon voi tallentaa tietoja nimi-arvo-pareina. Sama Bundle-olio tulee aktiviteetille onCreate() ja onRestoreInstanceState() -metodeille. Tiedon palautuksen voi tehdä kummassa tahansa näistä metodeista. Activity-luokka tarjoaa myös oletustoteutuksen onSaveInstanceData() ja onRestoreInstanceState()-metodeista, jotka osaavat monissa tapauksissa suorittaa tiedon tallennuksen ja palautuksen. Aktiviteetin tilanpalautusta tarvitaan usein, esimerkiksi aina kun käyttäjä vaihtaa sovelluksen suuntaa pysty- ja vaakasuuntien välillä.

Aktiviteettien vaihtumisen yhteydessä takaisinkutsujen järjestys on aina sama. Kun aktiviteetti A käynnistää aktiviteetti B:n, ensin kutsutaan aktiviteetti A:n onPause()-metodia, sitten aktiviteetti B:n onCreate(), onStart() ja onResume()-metodeita peräkkäin. Viimeiseksi kutsutaan aktiviteetti A:n onStop()-metodia, mikäli aktiviteetti B peittää sen kokonaan. Näin esimerkiksi aktiviteetti A:n onPause()-metodissa tietokantaan tallennetut tiedot ovat käytössä aktiviteetti B:tä käynnistettäessä. Jos muutoksia taas tekee onStop()-metodissa, ne tapahtuvat vasta aktiviteetti B:n käynnistyttyä.

\subsubsection*{Palat}

Androidin versiosta 3 (API-versio 11) asti on ollut mahdollista määritellä aktiviteetteihin paloja (Fragment-luokan aliluokkia). Palat ovat uudelleenkäytettäviä komponentteja, joita voi käyttää osana aktiviteetteja. Niiden avulla on helpompi luoda käyttöliittymiä, jotka skaalautuvat eri kokoisille näytöille. Isommalla näytöllä aktiviteetti voi pitää sisällään useita paloja, jotka pienemmällä näytöllä ovatkin omissa aktiviteeteissaan. Palojen elinkaari on riippuvainen siitä aktiviteetista, johon ne on sisällytetty. Kun aktiviteetti pysähtyy, niin pysähtyy myös aktiviteetin sisältämät palat. Samoin aktiviteetin tuhoutuessa tuhoutuvat myös palat. Aktiviteettien sisällä paloilla voi kuitenkin olla oma elinkaarensa, niitä voi käynnistää ja tuhota vapaasti aktiviteetin ollessa käynnissä.

Palojen elinkaarta hallitaan takaisinkutsumetodeilla, kuten aktiviteettejakin. Monet metoditkin ovat samoja, kuten onCreate(), onStart(), onPause() ja onStop(). Lisäksi paloilla on muutamia takaisinkutsumetoideita, joita aktiviteeteilla ei ole. onCreateView()-metodia kutsutaan, kun palan käyttöliittymä tulee ensimmäistä kertaa näkyviin käyttäjälle. onAttach()-metodia kutsutaan, kun pala on liitetty johonkin aktiviteettiin. Tällöin pala saa itselleen viitteen aktiviteettiin kommunikointia varten. onActivityCreated()-metodia kutsutaan, kun palan liittäneen aktiviteetin onCreated()-metodi on ajettu. onDestroyView()-metodia kutsutaan, kun palan käyttöliittymä tuhotaan, ja onDetach()-metodia kutsutaan kun palaan liitetty aktiviteetti irroitetaan palasta \cite{android}.

\subsection{Palvelut}

\begin{figure}[htb]
\includegraphics[width=100mm]{service_lifecycle.png}
\caption{Palvelun elinkaari} \label{service_lifecycle}
\end{figure}

Palvelut ovat pitkäkestoisia taustaoperaatioita. Muut sovelluskomponentit voivat käynnistää niitä, ja ne jatkuvat vaikka käyttäjä lopettaisi kyseisen sovelluksen käyttämisen. Palvelu voi esimerkiksi soittaa musiikkia, suorittaa verkkotransaktioita, kommunikoida sisällöntarjoajien kanssa tai tehdä levykirjoitusta.

Palvelut voivat olla kahdenlaisia. Käynnistettävät (\emph{started}) palvelut suorittavat tehtävänsä kun niiden startService()-metodia kutsutaan. Tällainen palvelu voi jatkaa pyörimistä taustalla, vaikka sovellus suljettaisiin. Tyypillisesti käynnistettävä palvelu tekee jonkin yhden operaation, kuten tiedoston latauksen tai lähettämisen, ja lopettaa sitten itsensä. Käynnistettävät palvelut eivät yleensä palauta palautusarvona kutsujalle mitään. Käynnistettävien palveluiden tulee sulkea itsensä operaation valmistuttua kutsumalla stopSelf()-metodia. Myös muut komponentit voivat sulkea palvelun kutsumalla stopService()-metodia.

Sidotut (\emph{bound}) palvelut ovat sellaisia, että sovelluskomponentit sitovat palvelun niihin kutsumalla bindService()-metoida. Sidotut palvelut tarjoavat asiakas-palvelin-rajapinnan sitovalle komponentille. Palvelu voi vastaanottaa pyyntöjä ja palauttaa vastauksia niihin. Palvelun elinkaari on sama kuin sen sitoneen komponentin. Useampi komponentti voi sitoa saman palvelun yhtä aikaa. Tällöin palvelu sulkeutuu kun viimeinenkin niistä lopettaa toimintansa. Sitominen vapautetaan kutsumalla unbindService()-metodia.

Useimmiten käynnistettävät ja sidotut palvelut ovat erillisiä, mutta joissain tilanteissa sama palvelu voi toimia sekä käynnistettävänä että sidottuna palveluna. Käynnistettäviä palveluita käytetään tyypillisesti pitkäkestoisiin taustaoperaatioihin, jotka suoritetaan taustalla ilman että käyttäjä puuttuu niiden toimintaan. Sidotut palvelut taas voivat tarjota sovellukselle minkä tahansa palvelurajapinnan, jonka kanssa sovellus voi kommunikoida palvelun elinkaaren ajan.

Palveluiden elinkaari on esitetty kuvassa \ref{service_lifecycle}. Aktiviteettien tavoin koko palvelun elinkaari tapahtuu onCreate() ja onDestroy()-kutsujen välissä ja palvelun alustus tapahtuu onCreate()-metodissa. Sidotun palvelun aktiivinen elinkaari on onBind() ja onUnbind()-kutsujen välillä. Käynnistettävän palvelun elinkaari puolestaan alkaa onStartCommand()-kutsusta kunnes se sulkee itsensä stopSelf()-kutsulla. onBind() ja onStartCommand() -metodit saavat parametrinaan aikeen, jonka niitä kutsunut komponentti antoi bindService() tai startService() -metodille \cite{android}.

\subsection{Sisällöntarjoajat}

Sisällöntarjoajat tarjoavat pääsyn pysyvästi tallennettuun tietoon. Ne kapsuloivat tiedon ja tarjoavat mekanismit tiedon yksityisyyden hallintaan. Sisällöntarjoajat toimivat rajapintana tiedon ja sovelluskoodin välillä. Kun sisällöntarjoajan tietoon halutaan päästä käsiksi, käytetään ContentResolver-oliota Context-luokassa, joka sitten kommunikoi itse sisällöntarjoajan kanssa.

Sisällöntarjoajat eivät ole välttämättömiä sovelluksessa, jos tietoon ei haluta päästä käsiksi muista kuin samasta sovelluksesta. Sovellustenväliseen kommunikointiin sisällöntarjoajat tarjoavat vakiorajapinnan, joka pitää huolen prosessienvälisestä kommunikoinnista ja tietoturvallisuudesta.

Androidin mukana tulee valmiiksi toteutetut sisällöntarjoajat esimerkiksi musiikille, videotiedostoille ja käyttäjän yhteystiedoille. Muutamia rajoitteita lukuunottamatta nämä sisällöntarjoajat ovat kaikkien sovellusten käytettävissä \cite{android}.

\subsection{Aikeet}
\label{intents}

Suurin osa Android-sovellusten kommunikaatiosta on tapahtumapohjaista. Niin aktiviteetit, palvelut kuin sisällöntarjoajatkin käynnistetään lähettämällä niille aie (\emph{intent}). Tapahtumia käytetään Androidissa, koska niiden avulla komponentit voidaan sitoa toisiinsa ajonaikaisesti ja vasta silloin, kun niitä varsinaisesti tarvitaan. Itse aie-oliot ovat passiivisia tietorakenteita, joissa on abstrakti kuvaus operaatiosta, joka halutaan suoritettavan, tai lähetysten (\emph{broadcast}) tapauksessa kuvaus siitä, mitä on tapahtunut. 

Aikeiden kohde voidaan nimetä ekspliittisesti ComponentName-kentässä. Tällöin annetaan kohdekomponentin täydellinen nimi paketteineen, jolloin kohde voidaan tunnistaa yksikäsitteisesti. Tämän muodon käyttäminen vaatii, että kutsuva komponentti tietää kohdekomponentin nimen. Sovelluksensisäisessä kommunikoinnissa tämä onnistuu, mutta sovellustenvälisessä kommunikoinnissa useinkaan ei. Tällöin kohde päätellään implisiittisesti muista aikeelle annetuista kentistä.

Action-kentässä annetaan tapahtuma, joka aikeella halutaan käynnistää, esimerkiksi puhelun aloitus, tai lähetysten vastaanottajien tapauksessa järjestelmässä tapahtunut tapahtuma, kuten varoitus akun loppumisesta. Intent-luokassa määritellään lukuisia vakioita erilaisia tapahtumia varten, mutta niiden lisäksi sovellukset voivat määritellä myös omia tapahtumia.

Data-kentässä annetaan tapahtumaan liittyvän tiedon osoite (URI) ja tyyppi (MIME). Näin vastaanottava komponentti tietää minkätyyppistä tietoa aikeeseen liittyy, ja mistä se löytyy. Category-kentässä kerrotaan, minkä tyyppisen komponentin odotetaan käsittelevän aikeen. Näitäkin Intent-luokka tarjoaa valmiita, mutta omien käyttö on mahdollista.

Aikeen vastaanottava komponentti voidaan päätellä kahdella tavalla. Komponentti valitaan ekplisiittisesti, jos ComponentName-kentässä on arvo. Tällöin muiden kenttien arvoista ei välitetä. Muussa tapauksessa Action-, Data- ja Category-kenttien arvojen perusteella selvitetään, mitä soveltuvia vastaanottavia komponentteja järjestelmään on asennettuna. Tässä käytetään apuna aiesuotimia (Intent filter).

Sovellukset voivat määritellä aiesuotimia, jotta järjestelmä tietää, mitkä sovellukset voivat ottaa vastaan aikeita. Aiesuotimet ovat komponenttikohtaisia, ja ne määrittelevät, mitä tapahtumia, tietotyyppejä ja kategorioita ne tukevat. Aiesuotimia käytetään hyväksi implisiittisessä kohteen määrittelyssä. Jos kohde on määritelty eksplisiittisesti komponentin nimellä, aiesuotimilla ei ole vaikutusta \cite{android}.

\subsection{Android manifest}

Jokaisella Android-sovellukseen kuuluu AndroidManifest.xml-tiedosto, joka sisältää järjestelmälle välttämätöntä tietoa sovelluksen ajamiseksi. Manifestissa määritellään sovelluksen Java-paketti, joka toimii samalla sovelluksen yksilöllisenä tunnisteena. Manifestissa on myös listattu sovelluksen komponentit, aktiviteetit, palvelut, sisällöntarjoajat ja lähetysten vastaanottajat, joista sovellus koostuu, sekä niiden toiminnallisuus ulkopuolelta tulevien aikeiden kannalta Lisäksi manifestissa ilmoitetaan, mitä oikeuksia sovellus tarvitsee toimiakseen sekä mitä oikeuksia toisilla sovelluksilla pitää olla, jotta ne voivat käyttää kyseisen sovelluksen palveluita. Näiden tietojen lisäksi manifestissa määritellään sovelluksen vaatimuksen ympäristöltään: mikä on sovelluksen vaatima Android APIn minimiversio sekä mitä kirjastoja sovellus tarvitsee toimiakseen. \cite{android}

Jos sovellus tarvitsee välttämättä laitteelta tiettyjä ominaisuuksia, on mahdollista suodattaa sovellus pois Androidin sovelluskaupan hauista, jos sitä haetaan laitteella, joka ei tue sovelluksen vaatimia ominaisuuksia. Tärkeimmät suotimet ovat androidin API:n minimiversio, tiettyjen lisälaitteiden olemassaolo ja näytön koko.

<uses-sdk>-direktiivillä (directive) määritellään Androidin APIn minimiversio. Jos laitteessa on käytössä pienempi API-versio kuin direktiivillä annettu, sovellusta ei näytetä hakutuloksissa. <support screens>-direktiivi määrittelee, millä näytön koolla sovellus toimii. Tavallisesti määrittelemällä jokin tuettu koko, sovelluskauppa olettaa, että laite tukee sen lisäksi myös isompia näyttökokoja, muttei pienempiä. On myös mahdollista määritellä erikseen kaikki tuetut näyttökoot.

<uses-feature>-direktiivillä voidaan määritellä mitä ominaisuuksia sovellus vaatii. Näitä on sekä laitteistotasolla, kuten kamera, kiihtyvyysanturi tai kompassi, että ohjelmistotasolla, kuten vaikka liikkuvat taustakuvat, joiden pyörittämiseen kaikissa Android-laitteissa ei riitä resursseja. <uses-feature>-direktiiviä käytetään, kun sovellus ei lainkaan toimi ilman kyseistä ominaisuutta \cite{android_compatibility}.
\clearpage
\section{Ohjelmistojen testaaminen ja mobiilisovellusten testaamisen erityispiirteitä}

Tässä luvussa esitellään ohjelmistojen testauksen peruskäsitteitä, testauksen asemaa ohjelmistotuotantoprosessissa, mobiilisovellusten testaamisen erityispiirteitä sekä pohditaan miten testaustyökaluja voi arvioida.

\subsection{Testaamisen peruskäsitteitä}

Testitapaus (test case) on yksittäinen testi, jolle on määritelty syötteet, suoritusehdot ja läpäisykriteerit. Testisarja (test suite) taas on joukko testitapauksia. Testisarja voi myös koostua useasta testisarjasta, jolloin esimerkiksi ohjelman jokaiselle komponentille voi olla oma testisarjansa ja yksi testisarja kattaa sitten kaikki yksittäisten komponenttien testisarjat \cite[153]{testing}.

Yksikkötestaus on useimmiten valkoinen laatikko -testausta (white box testing / structural testing / glass box testing), jolloin testejä voidaan kirjoittaa ohjelmakoodin perusteella. Testeiltä voidaan vaatia esimerkiksi tiettyä koodikattavuutta, jolloin varmistetaan, että mahdollisimman suuri osa ohjelmakoodista tulee suoritettua testien aikana \cite[154]{testing}.

Toiminnallisessa testauksessa (functional testing) testattavan ohjelman sisäistä rakennetta ei tunneta vaan ollaan kiinnostuttu vain syötteistä ja niitä vastaavista tulosteista. Toiminnallista testausta voidaan kutsua myös musta laatikko -testaukseksi (black box testing). Toiminnallisessa testauksessa ollaan kiinnostuttu ohjelman käyttäjälle näkyvästä toiminnasta ja niitä voidaan tehdä esimerkiksi ohjelman määrittelyn pohjalta \cite[161-162]{testing}.

Mallipohjainen testaus (model based testing) on toiminnallisen testauksen alalaji, jossa testitapaukset luodaan automaattisesti ohjelman spesifikaatiosta tehdystä mallista. Jos ohjelma mallinnetaan formaalilla mallilla, kuten äärellisellä automaatilla tai kieliopilla, voidaan testitapaukset generoida täysin automaattisesti. Semiformaaleilla menetelmillä, kuten luokka- tai oliokaavioilla, mallinnetuista ohjelmista generointi saattaa vaatia manuaalista työtä \cite[245-250]{testing}.

Jäljittely (mocking) on tekniikka, joka helpottaa yksikkötestien kirjoittamista. Yksikkötestien ulkoiset riippuvuudet voidaan korvata testeissä kontrolloitavilla jäljittelijäolioilla. Tällöin testit ovat vakaampia, koska ulkopuolisten komponenttien muokkaus ei vaikuta testeihin, testin haluttuun lopputulokseen vaikuttava ympäristö on helppo saada haluttuun muotoon. Tällöin testien on helppo testata myös sellaisia olosuhteita, jotka ovat harvinaisia, tai vaikea saada muokattua. Lisäksi jäljittelemällä voidaan korvata vielä tekemättömät ulkoiset riippuvuudet jäljittelijätoteutuksilla \cite{mocking}.

\subsection{Testaaminen osana ohjelmistotuotantoprosessia}

\begin{figure}[htb]
\includegraphics[width=130mm]{v_model.png}
\caption{Testauksen v-malli} \label{v_model}
\end{figure}

Ohjelmistotuotantoprosessia lähestytään yleensä jonkin elämänkaarimallin kautta. Testauksen näkökulmaa edustaa testauksen v-malli, joka on esitetty kuvassa \ref{v_model}. Mallin yleisessä versiossa lähdetään siitä, että jokaista ohjelmistotuotantoprosessin vaihetta vastaa jokin testausvaihe. Näin testaus tehdään ohjelmistotuotantoprosessissa näkyväksi ja yhtä tärkeäksi osaksi kuin itse sovelluksen kehittäminen. Kuvassa v:n vasemmalla haaralla on vesiputousmallin mukaiset ohjelmistotuotantoprosessin vaiheet. Ensimmäisenä vaiheena on vaatimusmäärittely. Tässä vaiheessa määritellään loppukäyttäjän tai asiakkaan tarpeet ja vaatimukset sovellukselle.  Viimeinen vaihe on alimpana oleva sovelluksen ohjelmointi. Oikealla haaralla on testausvaiheet, joissa jokaisessa varmistetaan, että ohjelma täyttää vasemman haaran samalla tasolla olevassa vaiheessa suunnitellut vaatimukset \cite[39-42]{testing_foundations}.

V-mallin alimmalla tasolla on komponentti- eli yksikkötestaus. Siinä testataan ohjelman pienempiä itsenäiseen toimintaan kykeneviä osasia, eli olio-ohjelmoinnin tapauksessa useimmiten luokkia. Android-sovellusten tapauksessa tällä voidaan ajatella yksittäistä Android-komponenttia, esimerkiksi aktiviteettia, tai jopa aktiviteettia pienempiä osia, kuten yksittäistä näkymä-luokkaa tai muuta toimintaa tarjoavaa alikomponenttia. Komponenttitestaus on useimmiten valkoinen laatikko -testausta ja sen suorittaminen vaatii ohjelmointitaitoa. Useimmiten sovelluksen ohjelmoijat suorittavat itse komponenttitestauksen. Komponenttitestauksella pyritään varmistamaan, että komponentti täyttää sen toiminnallisen määritelmän. Komponenttitestejä tehdään usein eri kielille kehitetyillä yksikkötestikehyksillä, kuten Javan JUnitilla. Oikean toiminnallisuuden testaamisen lisäksi on tärkeää testata myös toiminta väärillä syötteillä ja poikkeustilanteissa. Moderni tapa tehdä komponenttitestausta on kirjoittaa testikoodi ennen sovelluskoodia \cite[43-50]{testing_foundations}.

V-mallin seuraavalla tasolla on integraatiotestaus. Siinä vaiheessa oletetaan, että komponenttitestaus on jo tehty ja yksittäisten komponenttien omaan toimintaan liittyvät virheet on löydetty. Integraatiovaiheessa yksittäisten komponentit yhdistetään toisiinsa ja testataan, että niiden yhteistoiminta on oikeanlaista. Tavoitteena on löytää mahdolliset virheet komponenttien rajapinnoista ja komponenttien välisestä yhteistyöstä. Testauksessa voidaan käyttää apuna tynkiä (stub) sellaisista komponenteista, jotka eivät vielä ole valmiita \cite[50-52]{testing_foundations}.

V-mallin kolmannella testaustasolla on järjestelmätestaus (system testing). Järjestelmätestauksessa testataan, että täysin integroitu järjestelmä toimii kokonaisuutena kuten pitäisi. Alemmista tasoista poiketen järjestelmätestauksessa näkökulma on järjestelmän tulevan käyttäjän, kun alemmilla tasoilla testaus on luonteeltaan enemmän teknistä. Järjestelmätestauksessa varmistetaan, että järjestelmä kokonaisuudessaan täyttää sille asetetut toiminnalliset ja ei-toiminnalliset vaatimukset. Järjestelmätestauksessa ei tulisi käyttää enää tynkiä, vaan järjestelmän kaikki komponentit tulisi asentaa esimerkiksi erilliseen testiympäristöön testausta varten. Itse sovelluksen lisäksi järjestelmätestauksen piiriin kuuluu ohjelmiston dokumentaatio ja konfiguraatio \cite[58-61]{testing_foundations}.

V-mallin viimeinen vaihe on hyväksyntätestaus. Alemmilla tasoilla sovellus on ollut kehittäjän vastuulla, mutta hyväksyntätestauksessa järjestelmä testataan asiakkaan näkökulmasta. Tällöin pyritään varmistamaan, että tuotettu järjestelmä täyttää mahdollisen sopimuksen, että sen käyttäjät hyväksyvät sen käytön sekä mahdollisesti alfa tai beta -testauksen oikeilla järjestelmän loppukäyttäjillä.\cite[62-63]{testing_foundations}

Testauksen jokaisessa vaiheessa ollaan kiinnostuneita validoinnista ja verifioinnista. Validoinnissa varmistetaan, että toteutus täyttää järjestelmän alkuperäisen tehtävän, eli että rakennetaan oikeaa sovellusta. Verifioinnissa taas varmistetaan, että kehitysvaiheessa tuotettu sovelluksen osa vastaa sille tehtyä spesifikaatiota, eli että sovellus on tehty oikein. V-mallin eri vaiheissa validoinnin ja verifikaation painotukset vaihtelevat. Alemman tason testauksessa on kyse enemmän verifikaatiosta ja ylemmän tason testauksessa taas validoinnista \cite[41-42]{testing_foundations}.

\subsection{Mobiilisovellusten testaamisen erityispiirteitä}

Mobiilisovellusten kehittämiseen liittyy haasteita, jotka liittyvät mobiiliympäristön rajoituksiin. Näitä on muunmuassa päätelaitteiden rajallinen kapasiteetti ja jatkuva kehitys, erilaiset standardit, protokollat ja verkkotekniikat, päätelaitteiden monimuotoisuus, mobiililaitteiden käyttäjien erikoistarpeet sekä tiukat aikavaatimukset sovellusten saamiseksi markkinoille.

Sovelluksia rajoittaa myös mobiililaitteiden fyysinen koko ja laitteiden monimuotoinen koko, paino ja näytön koko. Myös käyttöliittymissä on eroja, joskin kosketusnäyttöpuhelimet ovat vallanneet suurimman osan markkinoista viime vuosina.

Mobiilisovelluksen on oltava laadultaan hyvä, jotta se on helppo saada toimimaan oikein erilaisissa laiteympäristöissä. Lisäksi julkaisunopeus voi olla kriittinen tekijä markkinaosuuden valtaamisessa. Jos kilpaileva sovellus julkaistaan viikkoa aikaisemmin, voi markkina olla jo täytetty sovelluksen julkaisuhetkellä.

VTT on kehittänyt mobiilisovellusten kehittämiseen ketterän Mobile-D-lähestymistavan. Testauksen kannalta olennaista on testilähtöisen kehityksen käyttäminen (test driven development, tdd). Mobile-D:hen kuuluu testien kirjoittaminen ennen tuotantokoodin kirjoitusta, yksikkötestien automatisointi ja kaikkien ominaisuuksien hyväksyntätestaus asiakkaan kanssa \cite{abrahamsson04}.

Testilähtöisessä kehityksessä on kaksi merkittävää periaatetta: ohjelmakoodia saa kirjoittaa vain automaattisen testin korjaamiseksi ja duplikaattikoodin poistamiseksi. Näistä periaatteista seuraa tunnettu tdd-sykli: punainen, vihreä, refaktoroi. Ensin kirjoitetaan testi, joka ei mene läpi, koska testin toteuttavaa ohjelmakoodia ei ole vielä olemassa. Vaiheen nimi on punainen, koska useimmilla yksikkötestityökaluilla lopputuloksena näkyy punainen palkki, jos jokin testi ei mene läpi. Toinen vaihe on kirjoittaa juuri sen verran koodia, mitä tarvitaan testin läpäisemiseksi. Tässä vaiheessa ei välitetä miten rumaa koodi on. Vaiheen nimi on vihreä, koska useimmillaa yksikkötestityökaluilla lopputuloksena näkyy vihreä palkki, kun testit menevät läpi. Viimeisessä vaiheessa refaktoroidaan toisessa vaiheessa mahdollisesti syntynyt duplikaatti- tai muuten ruma koodi. Testit auttavat varmistamaan, ettei refaktoroidessa hajoiteta vanhaa toiminnallisuutta. Jotta tällainen ohjelmointisykli olisi mahdollinen, ohjelmistoympäristön täytyy tarjota mahdollisuus saada nopeasti palaute pienestä testijoukosta, jottei ohjelmoidessa jouduta jatkuvasti odottamaan testien ajautumista \cite{tdd}.

\subsection{Testityökalujen arviointikriteereistä}

Androidille on tehty Androidin mukana tulevien testaustyökalujen lisäksi monia muita testaustyökaluja. Nämä työkalut erottautuvat Androidin työkaluista joko pyrkimällä toteuttamaan jonkin asian paremmin kuin vastaava Androidin oma testaustyökalu tai sitten tarjoamalla sellaisen lähestymistavan testaamiseen, mitä Androidin omat testaustyökalut eivät tarjoa.

Testaustyökalujen arviointia käsittelevät muunmuassa Poston ja Sexton \cite{poston92}. Hyvän testaustyökalun kriteerit ovat osin kontekstista riippuvia, mutta Poston ja Sexton määrittelevät myös yleisempiä kriteerejä testityökalujen arviointiin. Työkalun toiminnallisten ominaisuuksien arviointi on yleensä kontekstiriippuvaista, mutta usein myös helposti määriteltävissä. Jollain työkalulla pystyy testaamaan asioita, joita toisella ei voi. Ei-toiminnallisia olennaisia ominaisuuksia he luettelevat työkalun tehokkuuden, miten nopeasti sen käytön oppii, miten nopeaa testien tekeminen sillä on ja miten luotettava työkalu on.

Michael et al. \cite{michael02} ovat kehittäneet joukon metriikoita testityökalun tehokkuuden arviointiin. Heidän käyttämänsä metriikat ovat saaneet innoituksensa tavallisille sovelluksille kehitetyistä erilaisista kompleksisuusmetriikoista, kuten koodirivien määrä tai syklomaattinen kompleksisuus. He esittävät listan vastaavia metriikoita testityökalujen tehokkuuden arviointiin. Näitä arvoja voi sitten painottaa testityökalujen valinnassa haluamallaan tavalla. Osa metriikoista ei ole enää relevantteja nykyisten testaustyökalujen kannalta. Metriikoita on muunmuassa työkalun kypsyys ja käyttäjäkunnan koko, helppokäyttöisyys, kustomointimahdollisuudet, automaattinen testitapausten generointi, muiden ohjelmointityökalujen tarjoama tuki työkalulle, luotettavuus sekä suoritusnopeus. Osalle metriikoista tarjotaan täsmällisiä laskukaavoja, jotka helpottavat kvantitatiivisen analyysin tekemistä.


\clearpage
\section{Androidin mukana tulevista työkaluista}

Pitäiskö tässä luvussa olla jotain ylipäänsä testaamisen perusteita, vai riittääkö että johdannossa sivutaan aihetta?

\subsection{Android SDK:n mukana tulevia testityökaluja}

\begin{figure}[htb]
\includegraphics[width=100mm]{test_framework.png}
\caption{placeholder omalle kuvalle} \label{test_framework}
\end{figure}

Androidin SDK:n (suomennos!) mukana tulee monia testaustyökaluja. Testejä voi ajaa joko emulaattorilla tai suoraan puhelimessa.

Androidin testisarjat(suite?) perustuvat JUnit-työkaluun. Puhdasta JUnitia voi käyttää sellaisen koodin testaamiseen, joka ei kutsu Androidin rajapintaa tai sitten voi käyttää Androidin JUnit-laajennusta Android-komponenttien testaukseen. Laajennus tarjoaa komponenttikohtaiset test case (suomennos?) luokat. Nämä luokat tarjoavat apumetodeita mock-olioiden(suomennos?) luomiseen ja komponentin elinkaaren hallintaan. Androidin JUnit-toteutus mahdollistaa JUnitin versio 3:n mukaisen testityylin, ei uudempaa versio 4:n mukaista.

Kuvassa \ref{test_framework} on esitetty Androidin testien ajoympäristö. Testattavaa sovellusta testataan ajamalla testipaketissa olevat testitapaukset MonkeyRunnerilla (ks. luku \ref{monkeyrunner}). Testipaketti sisältää testitapausten lisäksi Androidin instrumentaatiota, eli apuvälineitä sovelluksen elinkaaren hallintaan ja koukkuja, joilla järjestelmän lähettämiä callback-metodikutsuja pääsee muokkaamaan, sekä mahdollisesti mock-olioita korvaamaan järjestelmän oikeita luokkia testin ajaksi mock-toteutuksella. \cite{android}

\subsection{Komponenttikohtaiset testiluokat}

Android tarjoaa aktiviteeteille, palveluille ja sisällöntarjoajille jokaiselle oman testiyläluokkansa, joka mahdollistaa komponenttikohtaisten testien helpomman toteutuksen.

Aktiviteettien testauksessa Androidin JUnit-laajennus on tärkeä, koska aktiviteeteilla on monimutkainen elinkaari, joka perustuu paljolti callback-metodeihin (suomennos), joiden suora kutsuminen ei ole mahdollista. Aktiviteettien testauksen pääyläluokka on InstrumentationTestCase. Sen avulla on mahdollista käynnistää, pysäyttää ja tuhota testattavana oleva aktiviteetti halutuissa kohdissa. Lisäksi sen avulla voi mockata järjestelmäolioita, kuten Contexteja ja Applicationseja. Tämä mahdollistaa testin eristämisen muusta järjestelmästä ja Intentien luomisen testiä varten. Lisäksi yliluokassa on metodit käyttäjäinteraktion, kuten kosketus- ja näppäimistötapahtumien lähettämiseen suoraan testattavalle luokalle.

Aktiviteettien testaamiseen on kaksi olennaista välitöntä yliluokkaa, ActivityUnitTestCase ja ActivityInstrumentationTestCase2. ActivityUnitTestCase on tarkoitettu luokan yksikkötestaamiseen siten, että se on eristetty Android-kirjastoista. Näitä testejä voi ajaa suoraan IDEstä ja tarvittaessa Android-kirjaston mockaamiseen on käytössä MockApplication-olio. ActivityInstrumentationTestCase2 taas on tarkoitettu toiminnalliseen testaukseen tai useamman aktiviteetin testaamiseen. Ne ajetaan normaalissa suoritusympäristössä emulaattorilla tai Android-laitteessa. Aikeiden mockaus on mahdollista, mutta testin eristäminen muusta tuotantojärjestelmästä ei ole mahdollista.

Palveluiden testaaminen on paljon yksinkertaisempaa kuin aktiviteettien. Ne toimivat eristyksessä muusta järjestelmästä, joten testattaessakaan ei tarvita Androidin instrumentaatiota. Android tarjoaa ServiceTestCase-yliluokan palveluiden testaamiseen. Se tarjoaa mock-oliot Application- ja Context-luokille, joten palvelun saa testattua eristettynä muusta järjestelmästä. Testiluokka käynnistää testattavan palvelun vasta kutsuttaessa sen startService() tai bindService()-metodia, jolloin mock-oliot voi alustaa ennen palvelun käynnistymistä. Mock-olioiden käyttö palveluiden testaamisessa paljastaa myös mahdolliset huomaamatta jääneet riippuvuudet muuhun järjestelmään, koska mock-oliot heittävät poikkeuksen, mikäli niihin tulee metodikutsu, johon ei ole varauduttu.

Sisällöntarjoajien testaaminen on erityisen tärkeää, jos sovellus tarjoaa sisällöntarjoajiaan muiden sovellusten käyttöön. Tällöin on myös olennaista testata niitä käyttäen samaa julkista rajapintaa, jota muut sovellukset joutuvat käyttämään kommunikoidessaan sisällöntarjoajien kanssa. Sisällöntarjoajien testauksen yliluokka on ProviderTestCase2, joka tarjoaa käyttöön mock-oliot ContentResolveriosta ja Contextista, jolloin sisällöntarjoajia voi testaja eristyksissä muusta sovelluksesta. Yliluokka tarjoaa myös metodit sovelluksen oikeuksien testaamisen. Contextin mock-olio mahdollistaa tiedosto- ja tietokantaoperaatiot, mutta muut Androidin kirjastokutsut on toteutettu stubeina. Lisäksi tiedon kirjoitusosoite on uniikki testissä, joten testien ajaminen ei yliaja varsinaista sovelluksen tallentamaa tietoa. Sisällöntarjoajatestit ajetaan emulaattorissa tai Android-laitteella. \cite{android}

\subsection{MonkeyRunner}
\label{monkeyrunner}

MonkeyRunner tarjoaa rajapinnan, jolla android-sovellusta voi ohjata laitteessa tai emulaattorissa. Se on lähinnä tarkoitettu toiminnallisten testien (functional tests, onko oikea suomennos?) sekä yksikkötestien ajamiseen, mutta soveltuu myös muihin tarkoituksiin. Sen avulla voi esimerkiksi asentaa sovelluksia, ajaa testisarjoja ja sovelluksia ja lähettää niihin syötteitä. Lisäksi monkeyrunnerilla voi ottaa eri kohdista kuvakaappauksia ja verrata niitä referenssikuviin. Tällä tavalla voidaan tehdä esimerkiksi regressiotestausta.

MonkeyRunnerilla voidaan testata yhtä aikaa esimerkiksi monia eri emulaattoreita tai useita laitteita, jolloin voidaan tehdä fragmentaatiotestausta. MonkeyRunner on myös laajennettavissa, jolloin sitä voi käyttää muihinkin tarkoituksiin. MonkeyRunneria ohjataan pythonilla ja se on toteutettu jythonilla, joka on Javan virtuaalikoneessa pyörivä python-toteutus.\cite{android}

\subsection{Uiautomator ja uiautomatorviewer}

Monkeyrunneria paremmat assert-mahdollisuudet ja mahdollisuuden toiminallisten mustalaatikko-testien kirjoittamiseen Javalla tarjoaa uiautomator. Siinä on kaksi komponenttia uiautomatorviewer ja uiautomator. Uiautomatorviewer on graafinen työkalu, jolla voi analysoida ja skannata Android-sovelluksen käyttöliittymää ja uiautomator työkalu, joka tarjoaa rajapinnan ja moottorin toiminnallisten mustalaatikkotestien ajamiseen.

\begin{figure}[htb]
\includegraphics[width=150mm]{uiautomatorviewer.png}
\caption{placeholder omalle kuvalle} \label{uiautomatorviewer}
\end{figure}

Uiautomatorviewer (kuvassa \ref{uiautomatorviewer}) blah

\cite{android}

\subsection{Monkey}

Monkey on Androidin mukana tuleva työkalu, jota voi ajaa emulaattorissa tai Android-laitteessa ja joka tuottaa pseudosatunnaisia syötteitä ohjelmalle, kuten painalluksia, eleitä sekä järjestelmätason viestejä. Monkeytä voi käyttää esimerkiksi sovelluksen stressitestaukseen tai fuzz-testaukseen.

Monkeylle voi antaa jonkin verran sen toimintaa ohjaavia parametreja. Ensinnäkin testisyötteiden määrää ja tiheyttä voi rajoittaa. Toiseksi erityyppisten syötteiden osuutta voi säätää. Kolmanneksi testauksen voi rajoittaa tiettyyn pakettiin sovelluksessa. Tällöin Monkey pysäyttää testauksen, jos se on ajautuu muihin kuin haluttuun osaan sovelluksesta. Neljänneksi Monkeyn tulosteiden määrää ja tarkkuutta voi säätää.

Monkey pystäyttää testin, jos ohjelmasta lentää käsittelemätön poikkeus tai jos järjestelmä lähettää sovellus ei vastaa -virheviestin. Näissä tapauksissa Monkey antaa raportin virheestä ja miten se syntyi. Monkey voi myös haluttaessa tehdä profilointiraportin testistä.\cite{android}

\subsection{Androidin fragmentaatio}

Google tarjoaa android-laitteiden fragmentaation seurantaan palvelua, jossa kerrotaan kahden viikon jaksolla Androidin sovelluskaupassa käyneiden laitteiden jakauma androidin version, näytön koon ja resoluution sekä OpenGL:n version mukaan.\cite{android_versions} Jakauma ei välttämättä vastaa käytössä olevien Android-laitteiden jakaumaa, mutta toisaalta sovelluskehittäjän kannalta ne käyttäjät, jotka käyttävät sovelluskauppaa, lienevät olennaisimpia.

\begin{table}[h]
\centering
\begin{tabular}{ l l l l }
  Versio & Koodinimi & API & Osuus \\
  1.5 & Cupcake & 3 & 0.1\% \\
  1.6 & Donut & 4 & 0.4\% \\
  2.1 & Eclair & 7 & 3.4\% \\
  2.2 & Froyo & 8 & 12.9\% \\
  2.3 - 2.3.2 & Gingerbread & 9 & 0.3\% \\
  2.3.3 - 2.3.7 & Gingerbread & 10 & 55.5\% \\
  3.1 & Honeycomb & 12 & 0.4\% \\
  3.2 & Honeycomb & 13 & 1.5\% \\
  4.0.3 - 4.0.4 & Ice Cream Sandwich & 15 & 23.7\% \\
  4.1 & Jelly Bean & 16 & 1.8\% \\
\end{tabular}
\caption{Androidin versioiden osuus 1.10.2012 päättyneellä 2-viikkoisjaksolla}
\label{tab:android_versions}
\end{table}

Taulukko \ref{tab:android_versions} kuvaa androidin käyttöjärjestelmäversioiden jakaumaa. Miksi olennainen? Androidin 4-versio julkaistiin lokakuussa 2011, mutta vuotta myöhemmin vain noin neljäsosa laitteista käyttää 4. versiota. Jos sovelluskehittäjä haluaa tukea esimerkiksi 90\% markkinoilla olevista laitteista, on tuki ulotettava 2.2-versioon, joka julkaistiin toukokuussa 2010. \cite{android_version_history}

\begin{table}[h]
\centering
\begin{tabular}{ l l l l l }
   & ldpi (~120dpi) & mdpi (~160dpi) & hdpi (~240dpi) & xhdpi (~320dpi) \\
  small & 1.7\% &  & 1.0\% &  \\
  normal & 0.4\% & 11\% & 50.1\% & 25.1\% \\
  large & 0.1\% & 2.4\% &  & 3.6\% \\
  xlarge &  & 4.6\% &  &  \\
\end{tabular}
\caption{Android-laitteiden näyttökokojen ja pikselitiheyden osuudet 1.10.2012 päättyneellä 7 päivän jaksolla.}
\label{tab:screen_sizes}
\end{table}

Android-laitteet poikkeavat toisistaan sekä näytön fyysisen koon, että pikselitiheyden puolesta. Taulukossa \ref{tab:screen_sizes} on kuvattuna erilaisten näyttötyyppien jakaumaa. Näytön koon arvioinnissa android käyttää tiheysnormalisoituja pikseleitä (Density-independent pixel, dp), jotka lasketaan kaavalla px = dp * (dpi / 160), missä px on pikseli ja dpi on pikseleiden määrä tuumalla. Siten esimerkiksi 240 dpi:n näytöllä, yksi dp vastaa puoltatoista fyysistä pikseliä. Sovellusten ulkoasu tulisi aina suunnitella käyttäen dp:tä yksikkönä, jolloin skaalaus eri kokoisille ja pikselitiheyksisille näytöille onnistuu parhaiten. Taulukossa \ref{tab:screen_sizes} näyttökoot on lajiteltu niin, että xlarge näyttöjen resoluutio on vähintään 960dp x 720dp, large: 640dp x 480dp, normal: 470dp x 320dp ja small vähintään 426dp x 320dp.

\begin{table}[h]
\centering
\begin{tabular}{ l l }
  OpenGL ES versio & jakauma \\
  1.1 & 9.2\% \\
  2.0 \& 1.1 & 90.8\% \\
\end{tabular}
\caption{OpenGL versiot}
\label{tab:opengl_versions}
\end{table}

Taulukossa \ref{tab:opengl_versions} on kuvattu OpenGL ES -versioiden jakauma Android-laitteissa.\cite{android_versions}

Oheisten muuttujien lisäksi Android-laitteet poikkeavat toisistaan myös monilla muilla tavoin. Suoritintehoa laitteissa on hyvin eri määrissä käytössä ja näytönohjainten tehotkin vaihtelevat. Lisäksi erilaisia lisälaitteita, kuten gps, kiihtyvyysantureita, kompasseja yms. saattaa olla laitteissa, tai olla olematta.

\subsection{Mitä keinoja android sdk tarjoaa fragmentaation hallintaan}

Jos sovellus tarvitsee välttämättä laitteelta tiettyjä ominaisuuksia, on mahdollista suodattaa sovellus pois Androidin sovelluskaupan hauista, jos sitä haetaan laitteella, joka ei tue sovelluksen vaatimia ominaisuuksia. Tärkeimmät suotimet ovat androidin API:n minimiversio, tiettyjen lisälaitteiden olemassaolo ja näytön koko.

<uses-sdk>-direktiivillä (directive suomennos?) määritellään Androidin APIn minimiversio. Jos laitteessa on käytössä pienempi API-versio kuin direktiivillä annettu, sovellusta ei näytetä hakutuloksissa. <support screens>-direktiivi määrittelee, millä näytön koolla sovellus toimii. Tavallisesti määrittelemällä jokin tuettu koko, sovelluskauppa olettaa, että laite tukee sen lisäksi myös isompia näyttökokoja, muttei pienempiä. On myös mahdollista määritellä erikseen kaikki tuetut näyttökoot.

<uses-feature>-direktiivillä voidaan määritellä mitä ominaisuuksia sovellus vaatii. Näitä on sekä laitteistotasolla, kuten kamera, kiihtyvyysanturi tai kompassi, että ohjelmistotasolla, kuten vaikka liikkuvat taustakuvat, joiden pyörittämiseen kaikissa Android-laitteissa ei riitä resursseja. <uses-feature>-direktiiviä käytetään, kun sovellus ei lainkaan toimi ilman kyseistä ominaisuutta. Jos sovellus on käyttökelpoinen myös ilman ominaisuutta, voi tämän hallintaan käyttää muita keinoja (ja ne oli..?)  \cite{android_compatibility}
\clearpage
\section{Yksikkötestityökaluja}

robolectric vs android test framework k

\subsection{Robolectric}

Robolectric on yksikkötestaustyökalu, jonka tarkoitus on mahdollistaa android-koodin yksikkötestaus suoraan ohjelmointiympäristössä Javan virtuaalikoneessa ilman emulaattoria. Tarkoitus on mahdollistaa nopea TDD-sykli ja helpompi integrointi jatkuvan integroinnin palveluihin. Normaalisti Android-kirjaston luokat palauttavat kutsuttaessa IDE:stä ajoaikaisen poikkeuksen, mutta Robolectric korvaa nämä luokat varjototeutuksilla, jotka palauttavat poikkeuksen sijaan tyhjän oletusvastauksen (kuten null, 0 tai false) tai jos Robolectricissa on kyseistä metodia varten olemassa varjototeutus, se palauttaa toteutuksen määrittelemän oikeamman paluuarvon.

Robolectricin varjoluokkien käytön vaihtoehtona on jonkin mock-kehyksen käyttäminen androidin kirjaston korvaamiseen, mutta tämä on hyvin työläs ja verboosi tapa kirjoittaa testejä. Lisäksi tällöin testejä kirjoittaessa täytyy tuntea testattavan metodin toiminta hyvin tarkasti, jotta mock-toteutukset saadaan kirjoitettua. Robolectricin varjoluokat mahdollistavat enemmän musta laatikko -tyyppisen testauksen. \cite{robolectric}

\subsection{Aiempaa tutkimusta}

Sadeh et al. vertasivat Androidin aktiviteettien yksikkötestausta JUnitilla, AIT:lla (lyhenne/suomennos?) ja Robolectricilla. JUnitilla testattaessa ongelmaksi muodostui se, että ohjelmakoodia jouduttiin melko rajusti refaktoroimaan(suomennos?), jotta luokkien yksikkötestaus onnistui. Tämä tekee ohjelman ylläpidon vaikeaksi. JUnitin hyvä puoli oli erittäin nopea testien ajonopeus. Robolectricillä testien tekeminen taas oli lähes yhtä helppoa kuin AIT:lla. AIT:hen verrattuna Robolectricin vahvuudet oli virhepaikkojen paikantamisen helppous ja nopea ajonopeus. Robolectric-testit ajautuivat viisi kertaa nopeammin kuin AIT:lla ajetut testit, koska AIT ajaa testit emulaattorilla Dalvikilla, Robolectric taas suoraan Javalla. AIT:n suurin vahvuus oli testien kirjoittamisen helppous.\cite{sadehetal11}

Sadeh et al. eivät käyttäneet JUnit-testeissään mitään mock-työkalua, joten testeissä testiluokan riippuvuudet mockattiin tekemällä niistä staattisia sisäluokkia testiluokan sisään. Tämä on erittäin verboosi tapa ja vaikutti osaltaan siihen, miksi puhdas JUnit-testi näytti niin hankalalta. Androidin kirjastoluokat on pakko eristää testattavasta luokasta Javan virtuaalikoneella testattaessa, koska androidin kehitysympäristössä käytettävä paketti ei sisällä varsinaisesti luokkien sisältöä, vaan vain niiden luokkien julkiset rajapinnat, jolloin kehitystyökalut osaavat auttaa niiden käytössä, mutta varsinaista toteutusta ei ole.

Jeon \& Foster mainitsevat Robolectricin vahvuudeksi sen, että se pyörii Javan virtuaalikoneessa ja näin ohittaa hitaan vaihene testeistä, kun sovellus pitää kääntää emulaattorille tai laitteelle tetattavaksi. Robolectric ei heidän mielestään kuitenkaan sovellu kokonaisten sovellusten testaamiseen, koska sen varjoluokat eivät toteuta Androidin komponenteista kuin osan. Ylipäänsä Robolectric vertautuu Jeonin ja Fosterin mielestä paljolti robotiumiin. \cite{troyd}

Allevato \& Edwards käyttivät Robolectriciä opetuskäyttöön tarkoitetun RoboLIFT-työkalun kehitykseen. Robolectric auttoi heitä ohittamaan emulaattorin käytön ja nopeuttamaan opiskelijoiden testisykliä ja automaattista arviointialgoritmia. Tässä käytössä Robolectricin ongelma oli, että se ohittaa käyttöliittymän piirtämisen kokonaan ja muunmuassa näkymien onDraw-metodia ei kutsuta ollenkaan. Tämän seurauksena esimerksi näkymän leveys on aina 0 pikseliä, jolloin sellaiset testit, joilla haluttiin klikata näytöllä johonkin suhteelliseen kohtaan (vaikkapa keskelle) eivät toimineet oikein. \cite{robolift}

\subsection{Testit}

Testit.

\subsection{Analyysi}

Se.
\clearpage
\section{Toiminnallinen testaus}

Robotium vs. monkeyrunner

\subsection{Robotium}

Robotium ei tule Androidin sdk:n mukana, mutta se on paljon käytetty testityökalu Android-sovellusten testauksessa. Robotiumin slogan on, että se on kuin Selenium, mutta Androidille. Selenium taas on laajasti integraatio- ja funktionaalisessa testauksessa käytetty työkalu, joka mahdollistaa selaimen toimintojen automatisoimisen, kuten linkkien klikkauksen, lomakekenttien täyttämisen jne. \cite{selenium}

Robotium on tarkoitettu Android-sovellusten funktionaaliseen-, systeemi- ja hyväksyntätestaukseen. Se on black box -työkalu, eli testin kirjoittajan ei tarvitse päästä käsiksi tai tuntea testattavan sovelluksen koodia. Robotium-testit voivat testata samassa testitapauksessa useita aktiviteetteja. Robotium-testeissä annetaan ohjeita, missä järjestyksessä käyttöliittymäelementtejä klikataan tai syötetään tekstiä.

Robotiumtestejä voi ajaa niin emulaattorissa kuin puhelimessakin. Testit eivät kuitenkaan voi käsitellä kahta eri sovellusta, eli yksi testitapaus voi käsitellä vain yhtä sovellusta. Tällöin sovellustenvälinen integraatiotestaus ei ole mahdollista.

Robotiumin sivuilla sille esitellään useita vahvuuksia Android SDK:n mukana tuleviin työkaluihin verrattuna. Testit vaativat vain vähäistä tuntemusta testattavasta sovelluksesta, Robotium tukee usean aktiviteetin testaamiseta samassa testissä, testien kirjoittamisen nopeus, testikoodin selkeys ja sitkeys, joka johtuu ajoaikaisesta sidonnasta käyttöliittymäkomponentteihin, nopea suoritusnopeus ja helppo integrointi jatkuvan integroinnin työkaluihin Antin tai Mavenin avulla. (pitäisikö ant/maven esitellä jossain?) \cite{robotium}

\subsection{Troyd}

Troyd on Robotiumia käyttäen tehty integraatiotestaustyökalu, jonka tavoite on yhdistää Monkeyn skriptausominaisuudet ja Robotiumin tarjoama korkean tason API. Troyd-testit käyttävät korkean tason komentoja, kuten paina nappia nimeltä x, tarkista, että ruudulla näkyy teksti y, jne, joten testien kirjoituksen pitäisi olla nopeaa. Lisäksi Troyd tarjoaa nauhoitus-toiminnon, jolla testiä voidaan kirjoittaa siten, että testiä kirjoittaessa ohjelma etenee aina seuraavaan tilaan testin mukaisesti. Lopuksi testi tallentuu testitapauksiksi. \cite{troyd}

Troyd-testejä kirjoitetaan Rubylla käyttäen Rubyn Test::Unit-työkalua, joka on Rubyn standardi yksikkötestityökalu. \cite{testunit} Troydin komennot sisältävä TroydCommands-moduli sisällytetään testiluokkaan käyttämällä Rubyn mixin-toiminnallisuutta. Testitapauksia voi kirjoittaa kuten tavallisia test::unit-testejä tai sitten voi käyttää rec-skriptin nauhoitusmahdollisuutta.

Troydin heikkouksia on Jeonin ja Fosterin mielestä mahdollisuus testata vain yhtä sovellusta kerrallaan. Esimerkiksi, jos sovellus aukaisee selainikkunan, Troyd menettää sovelluksen kontrollin. Tämä johtuu Androidin testi-instrumentaation rajoituksista. Toinen Troydin heikkous on hidas suoritusnopeus, koska testiskripti odottaa jokaisen komennon jälkeen, että sovellus on oikeassa tilassa ennen testin jatkamista. \cite{troyd}

\subsection{TEMA}

Pleh.

\subsection{Aiempaa tutkimusta}

Jeon \& Foster mainitsevat Robotiumin vahvuudeksi Androidin omaa Instrumentatiota rikkaamman APIn. Esimerkiksi nappien painamiseen voidaan käyttää nappien nimeä, josta Robotium laskee napin sijainnin. He myös vertaavat Robotiumia omaan Troyd-työkaluunsa ja sanovat sen heikkoudeksi, että testit pitää määritellä etukäteen, eikä niitä pysty muokkaamaan ajonaikaisesti. Muulta toiminnallisuudeltaan Troyd ja Robotium ovat suunnilleen samankaltaisia, koska Troyd on tehty Robotiumin päälle. \cite{troyd}

\subsection{Asennus}

\begin{lstlisting}[float,label=robotium_setup,caption=Robotium testirunko]
public class RobotiumTest extends ActivityInstrumentationTestCase2<Tomdroid> {

	private Solo solo;
	
	public RobotiumTest() {
		super(Tomdroid.class);
	}
	
	@Override
	public void setUp() {
		solo = new Solo(getInstrumentation(), getActivity());
	}
	
	@Override
	public void tearDown() throws Exception {
		solo.finishOpenedActivities();
	}
}
\end{lstlisting}

Robotiumin asennus on helppoa, katso listaus \ref{robotium_setup}

\subsection{Robotium-testit}

\begin{lstlisting}[float,label=robotium_createnote,caption=Muistikirjan luontitesti robotiumilla]
public void testCreateNoteAddsNote() {
	solo.assertCurrentActivity("Testi alkoi väärästä aktiviteetista", Tomdroid.class);
	assertFalse(solo.searchText("new note"));
	solo.clickOnActionBarItem(R.id.menuNew);
	solo.assertCurrentActivity("Uuden muistikirjan luonti ei avannut uutta muistikirjaa editointinäkymään", EditNote.class);
	solo.enterText(0, "new note");
	solo.clickOnActionBarItem(R.id.edit_note_save);
	solo.clickOnActionBarHomeButton();
	solo.assertCurrentActivity("Koti-näppäimen painaminen ei vienyt takaisin muistikirjalistaan", Tomdroid.class);
	assertTrue(solo.searchText("new note"));
}
\end{lstlisting}

Robotium-testi, jossa testataan uuden muistikirjan luonti, on esitetty listauksessa \ref{robotium_createnote}. Robotiumilla testiä ohjataan Solo-luokan instanssin kautta, jossa on sovelluksen kanssa kommunikointiin tarkoitettuja metodeja, sovelluksen tilasta kertovia metodeja, sekä assertteja. Testin ensimmäisellä rivillä käytetään assertCurrentActivity()-metodia asserttia varmistamaan, että testi alkaa muistikirjalistasta. Toisella rivillä varmistetaan, että testissä luotavaa muistikirjaa ei vielä löydy listasta. Ilman tätä testissä ei voisi olla varma, että muistikirja on luotu onnistuneesti juuri testin aikana. Seuraavalla rivillä painetaan yläpalkin uuden muistikirjan luovaa nappia clickOnActionBarItem()-metodilla. Se ottaa parametrina komponentin id:n, johon ollaan painamassa. Tämän jälkeen pitäisi avautua uusi muistikirja editointinäkymään, mikä varmistetaan seuraavalla rivillä. Sitten syötetään enterText()-metodilla uuden muistikirjan otsikoksi \"new note\". Ensimmäinen parametri kertoo, monenteenko ruudulla näkyvään tekstinmuokkauskomponenttiin teksti syötetään. Tämän jälkeen klikataan yläpalkin tallennus-nappia ja sitten muistikirjalistaukseen vievää nappia. Lopuksi vielä varmistetaan, että palattiin takaisin muistikirjalistaan ja listasta löytyy nyt juuri luotu aktiviteetti.

\subsection{Analyysi}

Se.
%%\clearpage
%%\section{Muita Android-testityökaluja}

\subsection{TEMA}

Yms.

\subsection{Fuzzing}

Sovellusten turvallisuusominaisuuksien testaaminen on vaikeaa, koska testitapa poikkeaa tavanomaisesta. Yleensä testeillä halutaan varmistaa, että sovelluksessa on jokin ominaisuus kun taas turvallisuustestaus on negatiivista testaamista: halutaan varmistaa, että sovelluksessa ei ole turvallisuusaukkoja tai muita ei-haluttuja ominaisuuksia. Tällöin on mahdotonta kirjoittaa testitapauksia, koska ei voida mitenkään testata kaikkia mahdollisia sovelluksen suorituspolkuja.\cite{mahmoodetal12}

Fuzzing
\clearpage
\section{Yhteenveto}

Android-sovellukset koostuvat Android-kohtaisista komponenteista: aktiviteeteista, palveluista ja sisällöntarjoajista, sekä tavallisista Java-luokista. Aktiviteetti kuvaa käyttäjälle näkyvää näkymää. Palvelut taas on tarkoitettu pitkäkestoisten taustaoperaatioiden suoritukseen. Sisällöntarjoajat tarjoavat rajapinnan sovelluksen tarvitsemaan tietoon ja mahdollistavat sovellustenvälisen tietojenvaihdon.

Androidin arkkitehtuuri on vahvasti tapahtumapohjaista. Aktiviteetit ja palvelut toteuttavat takaisinkutsumetodit komponenttien elinkaaren hallintaan. Komponentit eivät ole suoraan yhteydessä toisiinsa vaan niiden välillä kommunikoidaan tapahtumapohjaisilla aikeilla, jotka järjestelmä välittää vastaanottavalle komponentille.

Android-sovellusten testaamiseen on kehitetty runsaasti testaustyökaluja. Jo Androidin mukana tulevat työkalut tarjoavat varsin kattavan työkaluvalikoiman Android-sovellusten testaamiseen ohjelmistotuotantoprosessin eri vaiheissa. Tämän lisäksi kolmannen osapuolen kehittämät testaustyökalut, kuten Robolectric ja Robotium, täydentävät Googlen kehittämiä testaustyökaluja.

Tässä tutkielmassa vertailtiin Androidin omia ja kolmansien osapuolien yksikkö- ja toiminnallisen testauksen työkaluja.

Yksikkötestauksessa haasteita tuo se, että vaikka Android-sovelluksia ohjelmoidaan Javalla, Androidin kirjastoluokat eivät toimi suoraan Javan omassa virtuaalikoneessa, vaan testit on ajettava Dalvik-ajoympäristössä emulaattorilla tai Android-laitteessa. Tämä hidastaa testien ajamista. Robolectric-yksikkötestaustyökalu mahdollistaa yksikkötestien ajamisen suoraan Javan virtuaalikoneella. Robolectric osoittautui toimivaksi vaihtoehdoksi Androidin omalle yksikkötestityökalulle. Sen käyttö oli käytännössä yhtä helppoa ja testien ajoaika oli moninkerroin nopeampi kuin Androidin omien yksikkötestien emulaattorissa.

Toiminnallisen testauksen työkaluista vertailtiin Uiautomatoria, Robotiumia ja Troydia. Kukin näistä on toimintatavaltaan hieman erilainen. Uiautomator on Androidin mukana tuleva työkalu, mutta toimii vain Androidin versiolla 4.1 tai uudemmalla. Uiautomator-testien kirjoittaminen on melko verboosia, mutta siihen liittyvä käyttöliittymän analysoiva Uiautomatorviewer tekee testien kirjoittamisesta helppoa. Troyd taas toimii vain Androidin versiolla 2.3.6. Sen erikoisuus on nauhoitusskripti, jonka avulla testiä kirjoitettaessa näkee jatkuvasti sovelluksen tilan testin kussakin vaiheessa. Troyd kuitenkin tarvitsisi runsaasti viimeistelyä, jotta sitä voisi suositella varauksetta. Robotium osoittautui hyväksi toiminnallisen testauksen työkaluksi, joka toimii vakaasti Androidin eri versioilla, joskin testien kirjoittaminen oli hieman Uiautomatoria tai Troydia haastavampaa. Testien ajoaika oli Robotiumilla ja Uiautomatorilla suunnilleen yhtä nopeaa. Troydin testien ajoaika oli reilusti kahta muuta työkalua hitaampi.

Androidin testaustyökalujen tilanne kokonaisuudessaan on varsin hyvä. Google on ottanut testauksen huomioon Androidia kehittäessä ja tarjoaa Androidin kehitystyökalujen mukana melko kattavan paketin erilaisia testityökaluja ohjelmistotuotantoprosessin eri vaiheissa. Tämän lisäksi kolmansien osapuolien kehittämät testaustyökalut täydentävät Androidin omien testaustyökalujen aukkoja.

\clearpage
\bibliographystyle{tktl}
\bibliography{lahteet}

\lastpage

\appendices

\internalappendix{1}{Yksikkötesteissä testatun sovelluksen lähdekoodi}

\begin{lstlisting}[caption=GameActivity.java]

package com.example.demo;

import com.example.demo.GameView.GameState;

import android.app.Activity;
import android.content.Intent;
import android.os.Bundle;

public class GameActivity extends Activity implements OnGameEndEventListener {
  public final static String SCORE = "com.example.demo.SCORE";
  private GameView mGameView;

  @Override
  protected void onCreate(Bundle savedInstanceState) {
    super.onCreate(savedInstanceState);
    setContentView(R.layout.game_layout);
    mGameView = (GameView) findViewById(R.id.gameview);
    mGameView.setGameEndEventListener(this);
    mGameView.setState(GameState.RUNNING);
  }

  @Override
  protected void onPause() {
    mGameView.setState(GameState.PAUSED);
    super.onPause();
  }
	
  @Override
  protected void onStop() {
    mGameView.setState(GameState.PAUSED);
    super.onStop();
  }
	
  @Override
  protected void onRestart() {
    mGameView.setState(GameState.RUNNING);
    super.onRestart();
  }
	
  @Override
  protected void onResume() {
    mGameView.setState(GameState.RUNNING);
    super.onResume();
  }
	
  public void onGameEnd(double score) {
    Intent intent = new Intent(this, MainActivity.class);
    intent.putExtra(SCORE, String.valueOf(score));
    startActivity(intent);	  
  }
}
\end{lstlisting}

\begin{lstlisting}[caption=GameView.java]
package com.example.demo;

import java.util.ArrayList;
import java.util.List;
import java.util.Random;

import android.content.Context;
import android.graphics.Canvas;
import android.graphics.Paint;
import android.os.Handler;
import android.os.Message;
import android.util.AttributeSet;
import android.view.MotionEvent;
import android.view.View;

public class GameView extends View {
	
  public enum GameState {
    READY, RUNNING, PAUSED, LOST;
  }
	
  private RefreshHandler mRedrawHandler = new RefreshHandler();
  private long mMoveDelay = 50;
  private List<Circle> mCircles;
  private Circle mPlayer;
  private GameState mState;
  private GameClock mClock;
  private OnGameEndEventListener mOnGameEndEventListener;
	
  public GameView(Context context, AttributeSet attributes) {
    super(context, attributes);
    mClock = new GameClock();
    initPlayer();
    initCircles();
    mState = GameState.READY;
  }

  private void initPlayer() {
    Paint paint = new Paint();
    paint.setARGB(255, 255, 0, 0);
    mPlayer = new Circle(20, 100, 100, paint);
  }

  private void initCircles() {
    mCircles = new ArrayList<Circle>();
    Paint circlePaint = new Paint();
    circlePaint.setARGB(255, 0, 0, 255);
    Random random = new Random();
    int numCircles = 3+random.nextInt(3);
    for (int i=0;i<numCircles;i++) {
      int radius = random.nextInt(35)+5;
      int speedX = random.nextInt(20)-10;
      int speedY = random.nextInt(20)-10;
      Circle circle = new Circle(radius, 300, 550, speedX, speedY, 450, 700, circlePaint);
      mCircles.add(circle);
    }
  }
	
  public void setState(GameState newState) {
    GameState oldState = mState;
    mState = newState;
    if (oldState != GameState.RUNNING && newState == GameState.RUNNING) {
      mClock.start();
      update();
    }
    if (newState != GameState.RUNNING) {
      mClock.pause();
    }
    if (mState == GameState.LOST && mOnGameEndEventListener != null) {
      mOnGameEndEventListener.onGameEnd(mClock.getTime()/1000.0);
    }
  }
	
  public GameState getState() {
    return mState;
  }
	
  public void setGameEndEventListener(OnGameEndEventListener onGameEndEventListener) {
    mOnGameEndEventListener = onGameEndEventListener;
  }

  @Override
  protected void onDraw(Canvas canvas) {
    super.onDraw(canvas);
    canvas.drawCircle(mPlayer.getX(), mPlayer.getY(), mPlayer.getRadius(), mPlayer.getPaint());
    for (Circle circle : mCircles) {
      canvas.drawCircle(circle.getX(), circle.getY(), circle.getRadius(), circle.getPaint());
    }
  }
	
  protected void update() {
    if (mState == GameState.RUNNING) {
      for (Circle circle : mCircles) {
        circle.update();
        if (circle.collisionWith(mPlayer)) {
          setState(GameState.LOST);
        }
      }
      mRedrawHandler.sleep(mMoveDelay);
    }
  }

  @Override
  public boolean onTouchEvent(MotionEvent event) {
    if (mState == GameState.RUNNING) {
      mPlayer.setX((int)event.getX());
      mPlayer.setY((int)event.getY());
      invalidate();
    }
    return super.onTouchEvent(event);
  }
	
  class RefreshHandler extends Handler {
    @Override
    public void handleMessage(Message msg) {
      GameView.this.update();
      GameView.this.invalidate();
    }

    public void sleep(long delayMillis) {
      this.removeMessages(0);
      sendMessageDelayed(obtainMessage(0), delayMillis);
    }
  }
}
\end{lstlisting}

\begin{lstlisting}[caption=MainActivity.java]
package com.example.demo;

import android.app.Activity;
import android.content.Intent;
import android.os.Bundle;
import android.view.View;
import android.widget.TextView;

public class MainActivity extends Activity {

  @Override
  protected void onCreate(Bundle savedInstanceState) {
    super.onCreate(savedInstanceState);
    setContentView(R.layout.main_layout);
    showScore();
	}

  private void showScore() {
    Intent intent = getIntent();
    String message = intent.getStringExtra(GameActivity.SCORE);
    TextView scoreText = (TextView) findViewById(R.id.scoretext);
    if (message != null && !message.equals("")) {
      scoreText.setText("Selvisit "+message+" sekuntia!");
    }
  }
	
  public void startGame(View view) {
    Intent intent = new Intent(this, GameActivity.class);
    startActivity(intent);
  }
}
\end{lstlisting}

\begin{lstlisting}[caption=Circle.java]
package com.example.demo;

import android.graphics.Paint;

public class Circle {

  private final int mRadius;
  private int mPositionX;
  private int mPositionY;
  private int mSpeedX;
  private int mSpeedY;
  private final Paint mPaint;
  private final int mWorldWidth;
  private final int mWorldHeight;

  public Circle(final int radius, final int positionX, final int positionY, final int speedX, final int speedY, final int worldWidth, final int worldHeight, final Paint paint) {
    this.mRadius = radius;
    this.mPositionX = positionX;
    this.mPositionY = positionY;
    this.mSpeedX = speedX;
    this.mSpeedY = speedY;
    this.mWorldWidth = worldWidth;
    this.mWorldHeight = worldHeight;
    this.mPaint = paint;
  }
	
  public Circle(final int radius, final int positionX, final int positionY, Paint paint) {
    this(radius,positionX,positionY,0,0,0,0,paint);
  }
	
  public int getX() {
    return mPositionX;
  }

  public int getY() {
    return mPositionY;
  }
	
  public void setX(int x) {
    mPositionX = x;
  }
	
  public void setY(int y) {
    mPositionY = y;
  }
	
  public Paint getPaint() {
    return mPaint;
  }
	
  public int getRadius() {
    return mRadius;
  }
	
  public void update() {
    mPositionX += mSpeedX;
    if (mPositionX > mWorldWidth-mRadius || mPositionX < mRadius) {
      mSpeedX = -1 * mSpeedX;
    }
    mPositionY += mSpeedY;
    if (mPositionY > mWorldHeight - mRadius || mPositionY < mRadius) {
      mSpeedY = -1 * mSpeedY;
    }
  }
	
  public boolean collisionWith(Circle circle) {
    Double distance = Math.sqrt(Math.pow(circle.getX() - mPositionX, 2) + Math.pow(circle.getY()-mPositionY, 2));
    return distance < mRadius + circle.getRadius();
  }
}
\end{lstlisting}

\begin{lstlisting}[caption=GameClock.java]

package com.example.demo;

public class GameClock {

  private long mStartTime;
  private long mPausedAt;
  private boolean mRunning;
	
  public GameClock() {
    mRunning = false;
    mStartTime = 0;
    mPausedAt = 0;
  }
	
  public void start() {
    if (!mRunning) {
      mStartTime = System.currentTimeMillis() - mPausedAt;	
    }
    mRunning = true;
  }
	
  public void pause() {
    if (mRunning) {
      mPausedAt = System.currentTimeMillis()-mStartTime;
    }
    mRunning = false;
  }
	
  public long getTime() {
    if (mRunning) {
      return System.currentTimeMillis()-mStartTime;
    }
    return mPausedAt;
  }
}
\end{lstlisting}

\begin{lstlisting}[caption=OnGameEndEventListener.java]
package com.example.demo;

public interface OnGameEndEventListener {
  public void onGameEnd(double score);
}

\end{lstlisting}

\begin{lstlisting}[caption=game\_layout.xml]
<?xml version="1.0" encoding="utf-8"?>
<LinearLayout xmlns:android="http://schemas.android.com/apk/res/android"
  android:layout_width="match_parent"
  android:layout_height="match_parent">

  <com.example.demo.GameView
    android:id="@+id/gameview"
    android:layout_width="match_parent"
    android:layout_height="match_parent" />

</LinearLayout>

\end{lstlisting}

\begin{lstlisting}[caption=main\_layout.xml]
<?xml version="1.0" encoding="utf-8"?>
<LinearLayout xmlns:android="http://schemas.android.com/apk/res/android"
  xmlns:tools="http://schemas.android.com/tools"
  android:layout_width="match_parent"
  android:layout_height="match_parent"
  android:orientation="vertical">

  <Button
    android:layout_width="wrap_content"
    android:layout_height="wrap_content"
    android:text="@string/start_game"
    android:onClick="startGame" />
    
  <TextView
    android:id="@+id/scoretext"
    android:text=""
    android:layout_width="wrap_content"
    android:layout_height="wrap_content" />

</LinearLayout>
\end{lstlisting}

\begin{lstlisting}[caption=AndroidManifext.xml]
<?xml version="1.0" encoding="utf-8"?>
<manifest xmlns:android="http://schemas.android.com/apk/res/android"
  package="com.example.demo"
  android:versionCode="1"
  android:versionName="1.0" >

  <uses-sdk
    android:minSdkVersion="8"
    android:targetSdkVersion="15" />

  <application
    android:icon="@drawable/ic_launcher"
    android:label="@string/app_name"
    android:theme="@style/AppTheme" >

    <activity
      android:name=".MainActivity"
      android:label="@string/title_game" >
            
      <intent-filter>
        <action android:name="android.intent.action.MAIN" />
        <category android:name="android.intent.category.LAUNCHER" />
      </intent-filter>
    </activity>
        
    <activity
      android:name=".GameActivity"
      android:label="@string/title_game" >
    </activity>
  </application>
</manifest>
\end{lstlisting}
\lastappendixpage

\end{document}

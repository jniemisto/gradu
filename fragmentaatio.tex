\section{Fragmentaatio}

Android-alustan suurimpia haasteita ohjelmistokehityksen ja testaamisen kannalta on fragmentaatio.

Appthwack\cite{appthwack}

%%http://techcrunch.com/2012/10/07/5-ways-to-manage-app-development-on-the-android-platform-without-going-nuts/?utm_source=feedburner&utm_medium=feed&utm_campaign=Feed%3A+Techcrunch+%28TechCrunch%29&utm_content=Google+Reader

\subsection{Androidin fragmentaatio}

Google tarjoaa android-laitteiden fragmentaation seurantaan palvelua, jossa kerrotaan kahden viikon jaksolla Androidin sovelluskaupassa käyneiden laitteiden jakauma androidin version, näytön koon ja resoluution sekä OpenGL:n version mukaan.\cite{android_versions} Jakauma ei välttämättä vastaa käytössä olevien Android-laitteiden jakaumaa, mutta toisaalta sovelluskehittäjän kannalta ne käyttäjät, jotka käyttävät sovelluskauppaa, lienevät olennaisimpia.

\begin{table}[h]
\centering
\begin{tabular}{ l l l l }
  Versio & Koodinimi & API & Osuus \\
  1.5 & Cupcake & 3 & 0.1\% \\
  1.6 & Donut & 4 & 0.4\% \\
  2.1 & Eclair & 7 & 3.4\% \\
  2.2 & Froyo & 8 & 12.9\% \\
  2.3 - 2.3.2 & Gingerbread & 9 & 0.3\% \\
  2.3.3 - 2.3.7 & Gingerbread & 10 & 55.5\% \\
  3.1 & Honeycomb & 12 & 0.4\% \\
  3.2 & Honeycomb & 13 & 1.5\% \\
  4.0.3 - 4.0.4 & Ice Cream Sandwich & 15 & 23.7\% \\
  4.1 & Jelly Bean & 16 & 1.8\% \\
\end{tabular}
\caption{Androidin versioiden osuus 1.10.2012 päättyneellä 2-viikkoisjaksolla}
\label{tab:android_versions}
\end{table}

Taulukko \ref{tab:android_versions} kuvaa androidin käyttöjärjestelmäversioiden jakaumaa. Miksi olennainen? Androidin 4-versio julkaistiin lokakuussa 2011, mutta vuotta myöhemmin vain noin neljäsosa laitteista käyttää 4. versiota. Jos sovelluskehittäjä haluaa tukea esimerkiksi 90\% markkinoilla olevista laitteista, on tuki ulotettava 2.2-versioon, joka julkaistiin toukokuussa 2010. \cite{android_version_history}

\begin{table}[h]
\centering
\begin{tabular}{ l l l l l }
   & ldpi (~120dpi) & mdpi (~160dpi) & hdpi (~240dpi) & xhdpi (~320dpi) \\
  small & 1.7\% &  & 1.0\% &  \\
  normal & 0.4\% & 11\% & 50.1\% & 25.1\% \\
  large & 0.1\% & 2.4\% &  & 3.6\% \\
  xlarge &  & 4.6\% &  &  \\
\end{tabular}
\caption{Android-laitteiden näyttökokojen ja pikselitiheyden osuudet 1.10.2012 päättyneellä 7 päivän jaksolla.}
\label{tab:screen_sizes}
\end{table}

Android-laitteet poikkeavat toisistaan sekä näytön fyysisen koon, että pikselitiheyden puolesta. Taulukossa \ref{tab:screen_sizes} on kuvattuna erilaisten näyttötyyppien jakaumaa. Näytön koon arvioinnissa android käyttää tiheysnormalisoituja pikseleitä (Density-independent pixel, dp), jotka lasketaan kaavalla px = dp * (dpi / 160), missä px on pikseli ja dpi on pikseleiden määrä tuumalla. Siten esimerkiksi 240 dpi:n näytöllä, yksi dp vastaa puoltatoista fyysistä pikseliä. Sovellusten ulkoasu tulisi aina suunnitella käyttäen dp:tä yksikkönä, jolloin skaalaus eri kokoisille ja pikselitiheyksisille näytöille onnistuu parhaiten. Taulukossa \ref{tab:screen_sizes} näyttökoot on lajiteltu niin, että xlarge näyttöjen resoluutio on vähintään 960dp x 720dp, large: 640dp x 480dp, normal: 470dp x 320dp ja small vähintään 426dp x 320dp.

\begin{table}[h]
\centering
\begin{tabular}{ l l }
  OpenGL ES versio & jakauma \\
  1.1 & 9.2\% \\
  2.0 \& 1.1 & 90.8\% \\
\end{tabular}
\caption{OpenGL versiot}
\label{tab:opengl_versions}
\end{table}

Taulukossa \ref{tab:opengl_versions} on kuvattu OpenGL ES -versioiden jakauma Android-laitteissa.\cite{android_versions}

Oheisten muuttujien lisäksi Android-laitteet poikkeavat toisistaan myös monilla muilla tavoin. Suoritintehoa laitteissa on hyvin eri määrissä käytössä ja näytönohjainten tehotkin vaihtelevat. Lisäksi erilaisia lisälaitteita, kuten gps, kiihtyvyysantureita, kompasseja yms. saattaa olla laitteissa, tai olla olematta.

\subsection{Mitä keinoja android sdk tarjoaa fragmentaation hallintaan}

Jos sovellus tarvitsee välttämättä laitteelta tiettyjä ominaisuuksia, on mahdollista suodattaa sovellus pois Androidin sovelluskaupan hauista, jos sitä haetaan laitteella, joka ei tue sovelluksen vaatimia ominaisuuksia. Tärkeimmät suotimet ovat androidin API:n minimiversio, tiettyjen lisälaitteiden olemassaolo ja näytön koko.

<uses-sdk>-direktiivillä (directive suomennos?) määritellään Androidin APIn minimiversio. Jos laitteessa on käytössä pienempi API-versio kuin direktiivillä annettu, sovellusta ei näytetä hakutuloksissa. <support screens>-direktiivi määrittelee, millä näytön koolla sovellus toimii. Tavallisesti määrittelemällä jokin tuettu koko, sovelluskauppa olettaa, että laite tukee sen lisäksi myös isompia näyttökokoja, muttei pienempiä. On myös mahdollista määritellä erikseen kaikki tuetut näyttökoot.

<uses-feature>-direktiivillä voidaan määritellä mitä ominaisuuksia sovellus vaatii. Näitä on sekä laitteistotasolla, kuten kamera, kiihtyvyysanturi tai kompassi, että ohjelmistotasolla, kuten vaikka liikkuvat taustakuvat, joiden pyörittämiseen kaikissa Android-laitteissa ei riitä resursseja. <uses-feature>-direktiiviä käytetään, kun sovellus ei lainkaan toimi ilman kyseistä ominaisuutta. Jos sovellus on käyttökelpoinen myös ilman ominaisuutta, voi tämän hallintaan käyttää muita keinoja (ja ne oli..?)  \cite{android_compatibility}
